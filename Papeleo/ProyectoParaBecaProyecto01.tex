\documentclass[letterpaper, 10pt]{article}


\usepackage[spanish]{babel}

\title{Proyecto de Investigacion}
\author{Wilhelm P. K. Zapfe Zaldivar}


\begin{document}

\maketitle

El estudio de las estructuras neuronales tiene que enfocarse en varios niveles
de complejidad. La evidencia apunta a a que existen arreglos de neuronas
altamente conectadas que operan como microcircuitos con funciones bien
definidas y que estos a su vez se encuentran ordenados modularmente. 
Entre pares de neuronas se conocian, hasta hace poco, dos formas de
comunicacion: puramente electrica o a travez de un neurotransmisor quimico. 
Las investigaciones del Dr. Rafael Gutierrez han 
demostrado que existe al menos otro tipo de comunicacion entre ellas,
una forma mixta, y esto
abre la posibilidad de que los circuitos sean mucho mas complejos de lo 
previamente esperado. Un arreglo de neuronas podria comunicarse entre ellas 
de las tres formas y formar un circuito operacional bien definido.

Por otro lado tenemos modelos que simulan la actividad de grupos completamente
conectados de neuronas y ralamente conectados entre ellos que parecen
indicar la existencia de ciertas estructuras generales en los  niveles mas alto
de la red. Un estudio parece indicar que ese modelo es adecuado para
explicar fenomenos de sincronizacion (y posiblemente ataques epilepticos)
a travez de la topologia de esta red superior. 

Eso ultimo aun se encuentra lejos de poder resolver el problema
de entender los posibles circuitos que podrian estar relacionados
con funciones basicas del sistema nervioso. 
El laboratorio dirigido por el Dr. Gutierrez cuenta con el equipo y personal para 
poder llevar a cabo un estudio a ese nivel. El ejemplo particular
que se trabaja actualmente es la actividad electrica de pequeñas rebanadas
del hipocampo de la rata comun. El equipo nos permite obtener bastante mas finos
espacialmente y con una resolucion temporal del orden de varios kilohertz. 
Esto ha permitito establecer una serie de objetivos con miras a entender la relacion
funcional en las conexiones del Giro Dentado (GA) y la estructura llamada
Cuerno de Amon (CA). Dado que la estructura esta organizada ahi de forma 
escencialmente laminar, el estudio profundo de la actividad en rebanadas
podra dilucidar la topologia de la red, y eventualmente, 
la funcionalidad de las conexiones en el hipocampo. 


\end{document}
