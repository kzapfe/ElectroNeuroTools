\documentclass[letterpaper, 10pt]{article}

\usepackage[spanish]{babel}
\usepackage[utf8]{inputenc}

\title{Proyecto de Investigacion}
\author{Wilhelm P. K. Zapfe Zaldivar}


\begin{document}

\maketitle

La investigación de las estructuras neuronales actualmente
ha establecido varios niveles funcionales que parecen ser
comunes. Hay evidencia muy clara de que las neuronas se agrupan
en módulos relativamente pequeños y altamente conectados que funcionarían
como microcircuitos y éstos a su vez estarían ordenados
en redes poco conectadas. La topología de una y otra
estructura jerárquica daría función al circuito así establecido.
Los trabajos enfocados en descubrir esas estructuras son
en este momento investigación de punta.

El Dr. Rafael Gutiérrez ha establecido que existe, al menos, otro
tipo de comunicación frente a los dos ya conocidos, de corriente iónica
intercelular y de neurotransmisores químicos. Uno podría describirse como
una señal analógica o de valores continuos, mientras que el otro
es una señal binaria discreta. El tipo mixto muestra características
de ambas en las mediciones del potencial eléctrico intracelular. También
encontró    que los tres tipos de conexiones se dan entre el giro dentado
(GA) y la zona CA3 en el hipotálamo de las ratas comunes. Esta
información previa es de fundamental importancia para
poder descubrir las estructuras funcionales que podríamos
describir como microcircuitos en el hipotálamo.

Actualmente se dispone de la capacidad de recopilar datos de pequeñas
rebanadas \emph{in vitro} del hipotálamo de las ratas. El equipo y personal
experimental nos permite observar la actividad eléctrica de pequeñas 
láminas del cerebro y grabarlas con una densidad muy alta en el espacio y de
muy alta resolución en el tiempo. Con estos datos, y el conocimiento
de las huellas de los diferentes tipos de conexiones podemos, a través
de un análisis estadístico y causal, proponer posibles redes que
conecten ambas estructuras. 

Para poder establecer finalmente un buen modelo explicativo de red para
los circuitos así observados sería bueno conocer, al menos cualitativamente,
las propiedades de las redes posibles, entre ellas, cualidades topológicas
y de las posibles dinámicas que podrían generar. El trabajo que yo aportaría
al grupo sería en esta línea, aparte de apoyo para el procesamiento
computacional de los datos. Mis conocimientos previos
de sistemas dinámicos no lineales y caóticos y capacidad de seguir
argumentos matemáticos abstractos en teoría de redes me permitiría
establecer cualidades generales en las propuestas de redes que obtendríamos
del análisis de los datos experimentales.



\end{document}
