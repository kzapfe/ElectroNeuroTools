\documentclass{article}

\usepackage{amsmath}
%\usepackage{natbib}
%\usepackage{textcomp} %para poner algunos simbolos griegos en el texto

\begin{document}


\section{Current source density analysis for high density MEAs}

The Current Source Density (CSD) can be obtained
by numerical differentiation of the LFP. We detail the procedure here.
We use the standard definition for the CSD:
\begin{equation}
  I_m = -\nabla\cdot(\sigma\nabla V),
\end{equation}
where $I_m$ is the CSD, $\sigma$ is the conductivity tensor, and $V$ is the LFP. We take
also the homogeneous, ohmic, isotropic assumptions, so that we can obtain
 the CSD up to a multiplicative factor by:
\begin{equation}
  I_m/\sigma=-\nabla^2 V
\end{equation}

In the experiments presented here, the spatial density of available recording electrodes
permitted us to approximate the Laplacian operator by a direct finite difference
method. The most convenient approach is the one proposed by Lindberg \cite{Lindberg90},
which approximates rotational invariance:
\begin{equation}
  \nabla^2_{L}=\begin{pmatrix}
  0 & 1 & 0 \\
  1 & -4 & 0 \\
  0 & 1 & 0
\end{pmatrix}
+1/3
\begin{pmatrix}
  0.5 & 0 & 0.5 \\
  0 & -2 & 0 \\
  0.5 & 0 & 0.5
\end{pmatrix}
\end{equation}
The application of this operator is the convolution with the data matrix (frame by frame). To
approximate border values of the CSD we pad the LFP matrix  by duplicating the outer rows and columns. 

This sort of operators are extremely sensitive to hard edges. The  noise of the data can be perceived in the spatial domain as rapidly varying values. Therefore,a Gaussian smoothing before the Laplacian operator can help us reduce excessive amplifying of noise. Our Gaussian blur filter has a $\sigma$ value of 126 $\mu$m, i.e., the distance covered by three electrodes.

A saturated or otherwise failing electrode can produce a larger error on the CSD representation,
as it affects all of its neighbors with a spurious value, often too large.
Therefore, before we can obtain the CSDA with the Lindberg operator, we have to deal with those
electrodes.
We have set up a routine of detecting saturated electrodes using two criteria: an average value over
10 ms that is unnaturally high (we put it at 2000 $\mu$V) or a moving standard deviation close to zero
over a 5ms interval. If any of those criteria was meet we declared the electrode unreliable. This changes
from measurement to measurement, so, the set of unreliable electrodes is different and very small.
In the measurements presented here, we always choose sets of data that reported less than 8 unreliable
electrodes, outside the main areas of interest.

Once that the set of failing electrodes is established, we simply discarded their data and replaced
it with the average of its 8 neighbors. Then we proceeded to apply the Gaussian blur filter and the Lindberg
Laplacian Operator. For graphical purposes we used the ``sinc''interpolation
routine included in the Matplotlib library \cite{Matplotlib} in order to dull the hard edges. 


\bibliographystyle{apalike}

\bibliography{../Reportes/BiblioReportes01.bib}






\end{document}
