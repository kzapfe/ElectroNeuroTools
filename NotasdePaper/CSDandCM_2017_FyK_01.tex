\documentclass{article}

\usepackage[uf8]{inputenc}
\usepackage{graphicx}

\title{Center of Mass dynamics in the CSD representation: a tool to convey activity in non convex neuronal tissue}

\authors{FK}

\begin{document}

\section{Introduction}

In previous work, researchers have been using a ``Center of Mass'' formalism to
track neuronal activity in various cases, as in the brain or the spinal tab
\cite{Manjarrez07, Majarrez09}.
The idea is simple enough: to mark a putative center of neural activity and
track its displacements as more or less units are firing.
The idea of center of mass is taken from classical mechanics, and it is a
nickname for a vectorial average.
The researchers have been using as vectors the location of the recording electrodes in an experiment, and as masses (or more adecuately, as density of mass)
various empirical measures that serve their purposes.

It is our aim in this paper to establish a mathematically adecuate framework
to this kind of analysis and to exemplify it with data from various experiments.
For neurophysiological interpretation see the upcoming paper by Gutierrez et al.


\section{Mathematical Formalism}

The most problematic issue with the ``masses'' used in previous works is
that they do not follow a precise definition of density.
Electrophysiological recordings have a natural representation using
a strict density: the Current Source Density or equivalently, the charge
density. The use of this representation has been forwarded as by
citing the advantages of localibility, and of sepparability.

CSD is usually obtained through numericall diferentiation, although
other more sofisticated techniques are beging to be used, notoriously
the inverse methods proposed by Potworowsky et all.

The idea is that, charge density being proportional to the CSD, we can
pinpoint exactly what the neurons are doing at a precise instant in time,
with better spatial resolution. They are expelling or absorving
positive metalic ions from the surrounding or the are at rest. It must be stated
that a single neuron can be engaged in both activities at different locations
of its body.




\end{document}
