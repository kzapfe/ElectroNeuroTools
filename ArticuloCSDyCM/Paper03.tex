\documentclass[12pt]{article}

\usepackage{amsmath}
\usepackage[utf8]{inputenc}
\usepackage[margin=1in]{geometry}
\usepackage{lscape}
\usepackage{mathptmx}
\usepackage[scaled=.90]{helvet}
\usepackage{courier}

\usepackage{setspace}
\usepackage{lineno} %ambos paquetes para borrador

\usepackage{graphicx}
\usepackage{longtable}

\usepackage{siunitx} % para unidades chidas como micrometros.


\doublespacing
\linenumbers

\DeclareMathOperator{\arcsinh}{arcsinh}
\newcommand{\mum}[1]{\SI{#1}{\micro\metre}}
\newcommand{\rd}{\mathrm{d}}


\author{WPK Zapfe and the Laboratorio 19 Crew}
\title{CSD, Disjoint components and trayectory detection}

\begin{document}

\maketitle

\section{Introduction}

The spatial tracking activity in neural tissue poses chalenging problems,
if one requires rigour, precision and clear interpretability. 
Measurement recordings
are usually presented  as a series of traces that correspond each to a sampling
electrode on the tissue. Then a series of causality tests are made on the
signal to search for possible efective conections. 
As Micro Electrode Arrays (MEAs) become denser, that representation
of the data saturates itself  and the discovery of patterns becomes a difficult task.
Novel information is obfuscated by the same high density acquisition that allows us to
discover it. Representing  the recorded data in ways which can unveil previously
invisible patterns is a necessity. Various techniques for separation of functional
generators are available now, most derived from the application of blind source
separation analysis, such as ICA. 
We present here  a method for tracing out
``functional local connectivity'', which would bring independent information
that is complementary to blind source separation.
Our  method would be applicable mostly to
activity in highly structured neural tissue. We shall exemplify it by
the data obtained from recordings of the rats hippocampus,
taking advantage of its highly stratified  structure. 

Local field potential (LFP)
can be regarded as generated by local multiunitary activity.
A patch of simultaneous, highly correlated, nearby traces
can be regarded as one of those multiunitary elements. 
Succesive activation  of distinct neighbouring elements on structured
tissue would trace a
``trajectory'' of the activity. A tracing of such trajectories would help to indicate
causal or correlated ``functional local conections'' over the tissue. Neighbouring
neurons that fire in succesion, integrated subthreshold activity and ephaptic coupling
could all be encompased as pathways for transmission of activity. Nevertheless,
how to define and acquire the putative paths is an open question.

In previous works some form of vector averaging has been proposed and shown to reveal some trajectories drawn by the activity. We shall call such methods Center of Mass Analyses (CMA).  The ``wheights'' that have been chosen for obtaining the average have been selected on heuristic grounds. In the works of Chao \emph{et al.} \cite{Chao05, Chao07}, the density of spikes in a certain time interval was used, a sort of ``averaged density of action'', while Manjarrez \emph{et al.} \cite{Manjarrez07, Manjarrez09} used the positive half of the LFP recording as their weight. 
Here we advance on such methods by incorporating mathematically sound definitions and concepts which allow us to rigorously separate spatially distinct traces. Our weight is actually the natural density function provided by the data: the Current Source Density. This representation not only provides a mathematical rigorous  measure for the weights in the vector averaging process, but helps to separate the data into disjoint sets and disconnected components.
This allows us to track separate process in the same region of interest
(r.o.i.) by following sepparately the distinct poles of action: influx and
eflux of charged ions. 

The application of the method requires electro-physiological data acquired
through bi-dimensional MEAs, although the extension to three dimensional arrays would be possible in principle. The MEA should have dense enough spatial and temporal samplings, so as to obtain good estimates of the CSD and their changes in time.

Our method permits us to map possible trajectories of activity inside structured neuronal tissue, without requiring ad hoc geometric constraints. As a prime example, non-convex structures which may show simultaneous activity in separate parts could show the robustness of our method. We have tested then the method on slices of rat's hippocampus, concentrating in the CA3 region. We exemplify the usage of the method with data that has been obtained on experiments on rat's hippocampal slices that are centered on the GD and CA
regions. The slices are put over an array of 64 by 64 electrodes
(BioCAM 4096 by 3Brain \cite{BioCam}),
and data is recorded at sampling rates between 7 and 14 kHz. The richness of the
data permits us to obtain a prime use case for our method. A
 certain amount of downscaling would be 
possible, but the array of electrodes should be at least 2 dimensional and
a interelectrode distance of the order of magnitude of a typical pyramidal soma
(around \mum{100}). 




\section{Methods and Calculations}

\subsection{Preprocessing}

After aquiring data from experiment, we pass the data through some preprocessing
routines. These serve to detect and discard unusable channels,
such as saturated or failling electrodes. The
set of failling electrodes varies from data set to data set,
as they can be produced by bubbles or by saturation after electrical stimulation
or other causes. 
We call this step the elimination of bad channels. We use a simple criterion for this:
if the absolute value of the signal is above certain threshold (which must be
adjusted for evocated vs spontaneous acttivity) or if the standard deviation is
larger than certain heuristically determined value, the channel is discarded.



\subsection{CSD for high density MEAS}

The concepts presented here require first obtaining the CSD from the recorded LFP. Any method would present the same utility, we have used two different
approaches. The first method is a  finite difference operator, while the second is the kCSDA presented by Potworowski \emph{et all} \cite{Potworowski2011}, which is an inverse problem approach.  Where both methods are usable, we prefer the former,
as it is faster and requires less assumptions.
We have used kCSDA where
the difference method is not applicable. The results are
independent of the method except for a multiplicative factor which
can be normalized.

Thanks to the density and scales of the BioCAM 4096,
the numerical finite difference method
produces an excelent approximation to the CSD,
if the number of failing channels is low or
they are outside the r.o.i.
In the experiments that were made without electrical stimulation that was the case,
with the order of magnitude of unusable channels in the tens.
A  numerical Laplacian Operator which reduces the cross effect of rectangular grids is the convolution of the data with the following matrix \cite{Lindberg90}:

\begin{equation}
\nabla^2_{1/3}=2/3
\begin{pmatrix}
  0 & 1 & 0 \\
  1 & -4 & 1 \\
  0 & 1 & 0
\end{pmatrix}
+1/3
\begin{pmatrix}
  0.5 & 0 & 0.5 \\
  0 & -2 & 0 \\
  0.5 & 0 & 0.5
\end{pmatrix}  
\end{equation}
This sort of operators are very sensitive to abrupt changes in the values.
The noise of the data can be perceived in the spatial domain as rapidly varying edges. A Gaussian spatial smoothing
can be performed before or after the Laplacian operator to the data in order to reduce excessive borders. Our Gaussian blur filter has a $ \sigma$ value of \mum{63}, which corresponds to one and a half times the interelectrode distance.
Intuitively this means that it smooths over each structures smaller than the soma of a typical pyramidal cell. 
This is  in accordance to our mean field approach, where we expect smooth differentiable fields over the scale of the neurons, but not in finer scales (for details see \cite{Bedard11}).

On previous works it has been argued that numerical difference operators sacrifice all data on the borders of the array. For electrode arrays of lesser density, this could posse a problem, but in our case we renounce to 253 out of 4095 channels (one is grounded), and those are outside the region of interest.
If very few electrodes fail, more sophisticated methods such as iCSDA \cite{Leski2011} and kCSDA \cite{Potworowski2011} prove little advantage in estimating the sources under the ohmic, isotropic and homogeneous assumptions. In case that around 30 or more electrodes
fail, specially if they are on the region of interest, the use of such methods
becomes necessary. The kCSDA method relies on the
assumption that a finite family of convenient functions can span, with acceptable accuracy, a very good aproximation to the  CSD at each instant in time.
A general presentation is in the reference \cite{Potworowski2011},we shall recount only the basic implementation as we used it. The
method can be summarized into two steps:
\begin{enumerate}
\item One presents a family of model functions for the CSD and
obtains the LFP that such functions
would generate.
\item Then one procedess to calculate
  the coefficients for the entire family by projecting the
  experimental data over this ideal LFP model function family.
\end{enumerate}

We shall borrow the notation in the original paper \cite{Potworowski2011}. Our model function for the CSD is a hard disk function, described by:
\begin{equation}
  \tilde{b}_j(x,y)=\begin{cases}
  1, \text{ for } (x-x_j)^2+(y-y_j)^2 \leq R^2 \\
  0 \text{ otherwise.} 
  \end{cases}
\end{equation}
The $R$ parameter is the effective radius of the charge density, which for our
calculations was taken as half the inter-electrode distance, that is $R=$\mum{21}.
Each pair $(x_j, y_j)$ is the center of one such model function, which we take as
the position of each  functioning electrode in an adequate coordinate system.
The LFP generated by this function would then be:
\begin{equation}
  \tilde{b}_j(x,y)=\frac{1}{2\pi\sigma}
  \iint \! \rd x \rd y \arcsinh \biggr(
  \frac{2 h } {\sqrt{(x-x_j)^2+(y-y_j)^2}}
  \biggl)
  \tilde{b}_j(x,y),
\end{equation}
  with $h$ being the effective height of the function support, in our case,
  the half width of the hippocampus slice ($h=$\mum{150}).

Then the kCSDA consist of projecting the experimental data (LFP), frame by frame,
over these model functions and obtaining the  coefficients that would spawn an idealized
LFP approximation. These coefficients are the same that would produce the
corresponding CSD. The first operator is the projection operator $K$ of the experimental
LFP (from now on $V(x,y)$) on the space of the idealized LFP (which we call $V^* (x,y)$).
\begin{equation}
  K_{jk}=\sum_{l=1}^{M}b_l(x_j,y_j)b_l(x_k,y_k).
\end{equation}
The sum is done over the set of $M$ electrodes that we want to use.
The other operator is the one that maps the space of $V^*(x,y)$ into the idealized
CSD functions, now called $C^*(x,y)$. 
\begin{equation}
  \tilde{K}_{jk}=\sum_{l=1}^Mb_l(x_j,y_j)\tilde{b}_l(x_k,y_k).
\end{equation}
Then $C^*(x,y)$ can be obtained on the region of interest by:
\begin{equation}
  C^*(x_j,y_j)=\sum_{k,l=1}^M \tilde{K}_{kj} K^{-1}_{ml} V(x_l,y_l)
\end{equation}
Notice how the method incorporates a denoising mechanism, so
no smoothing procedures are needed.
The great advantage of this method is that is possible to obtain a reasonable
approximation to CSD even when the presence of unusable channels is near or inside
the region of interest. The difference method, on the other hand, affects all surrounding
channels to a failing channel, up to 8 of them, so it becomes unusable if enough
saturated electrodes are over the r.o.i.

It is important to mention that our data has noise up to 0.1 of the mean magnitude of the
signal, which means that regularization of the matrix $K$ is unavoidable (see section 4.1 in \cite{Potworowski2011} and references therein). We used a regularization parameter $\lambda=0.001 |e_{max}|$, where $e_{max}$ is the principal eigenvalue of $K$. 

The attainment of $K$ and $\tilde{K}$ turns out to be computer-costly for high density MEAs, but paraletization, specially on GPUs, solves this issue.

%% posiblemente quitar esto 
We have performed comparison of both methods over experimental data that has
no failing channels in order to adjust the parameters of the second method. We
have obtained consistent results between them.



\subsection{CSD as weight density for vectors}


Center of Mass Analysis (CMA) has been used previously in order to track putative  ``trajectories'' of the activity or information across neural tissue or even the whole brain \cite{Chao05, Chao07, Manjarrez07, Manjarrez09}. In such works the chosen ``mass'' was an heuristic measure adecuate for representing the corse-grain picture of displacement of activity. These present the problem of justifing the choice made on practical grounds. For calculating a vectorial average, the adecuate wheight per vector is a density, a positive quantity that measures the concentration of something per space unit, so it is only adequate that we search for a natural density in the electrophysiological experiments. The Current Source Density fits the mathematical and interpretative requirements. We shall denote it in the following as $C$ (following \cite{Potworowski2011}).  This measure is usually presented in arbitrary units, which permits us to interpret it two-fold. One, directly from its definition, as the \emph{differential divergence} of the vectorial electric current density:
\begin{equation}
  C:=\nabla \cdot J
\end{equation}
Due to conservation of charges, this is the same as as the temporal variation of density of charge:
\begin{equation}
  C=-\frac{\partial \rho}{\partial t}
\end{equation}
Current Source Density as a mathematical density (a quasi-probability density function)  allows  to rigorously apply the concepts of vector averaging. This use has been overlooked. This might have occurred because  samplings at lesser spatial resolution did not permit estimations of this quantity with enough precision to make reliable spatial implications, even with the use of sophisticated CSD estimation methods. In our case, we want to use the density as a mathematician would: a function which assigns weights to points in the space of interest and permits us to carry linear operations over them.  The space here is the sampling points of the electrodes at a given instant in time, and the weight, the CSD, is amount of interchange of charged ions from the inside to the outside of cellular membranes, per unit space.  This is an indicator of activity in the neurons, therefore the CSD is also, indirectly, a \emph{``density of local activity''}. Having both
the mathematician and electrophysiologicist interpretations presents we can
perform the following analysis with clear readings. 

Apparently, CSD would not fit the requirements of non-negativity for its use as a mathematical density, but this is not a problem.  CSD separates the measured activity into three distinct sets, namely, the set of all sources, the set of all sinks and the zero set. The separation is not quantitative but qualitative: even when the signs of electric charge are arbitrary, the distinction between them is not. Therefore, CSD represents on each of the two active sets (sinks and sources) exactly what we need, a unit of concentration of changing charges per unit of space (in this case, per area), and the sign can be ignored, as long as we perform the operations in each set separately. Also, the separation by the zero value set of the CSD is in contrast to the zero value of the LFP, where it has no precise meaning as it only represents a practical reference value. This is acknowledged in most of the literature, where only the scales of the LFP recording are shown, without reference to the sign of the values. In the CSD representation,  the so called zero set (all the points in our space which have a recorded CSD of zero)  has a precise meaning: it is an instantaneous lack of activity and the border between sinks and sources. This shall be exploited further in the section below.

Once the separation into sinks and sources has been made, we would have two densities, or more precisely, two different density functions defined over two separate sets. Trying to use this for obtaining vector averages would still be crude and inexact. If these sets spread out over large non-convex areas, the vector averages could lie outside the sets, having little or none physiological interpretation. Therefore we decided to introduce another concept from geometrical sets in order to reduce the analysis to local effects.


\subsection{CSD and Disconnected Components}

A first visual inspection of the CSD color map at a given instant in time shows that both sources and sinks appear in rather large and significant patches, with typical
cross lengths around \mum{150}. .

These appear to correspond to the expected poles of firing neurons in structured tissue \cite{Buzsaki2012} and inspired us to separate the sink and source  sets into their \emph{disconnected components}. This concept is used in mathematics to denote subsets of a given geometric set that do not touch each other  \cite{Halmos}. Their rough correspondence to poles of firing neurons is most useful in tracking the successive locus of the activity.

To clarify the concept and how we are we using it an example is in order. In figure \ref{disconnectedsets} A we show geometrical sets enclosed by simple curves as color patches, where every color indicates a Set. The set $A$ consists of a single connected component, that means that one can make  path between any two points in the set that consist entirely of points of the same set. In contrast, the set $B$ is made of two disconnected components, labeled $B1$ and $B2$. If one chooses a point in each of the components, one cannot draw a path joining them without leaving the set. But each one of those components is connected, meaning that if we choose the pair of points in one component, then again we can trace a path joining them in the set. So the set $B$ consists of two disconnected components, but each component can be regarded as a connected set on their own. Likewise the set $C$ is made of three disconnected components. In the subfigure \ref{disconnectedsets} Aiii we show the center of mass of the sets. The set $A$ is concave, so, it can have a center of mass outside the set. The
set $B$ has also a center of mass outside the set. It has two disconnected components that are aproximatelly
equal and separated by a long distance (this is not an implication: a set can be disconnected and have
a CM inside the set). For the set $C$ our aproach has been used: we treat every disconnected component of the
set as an independent entity and calculate the center of mass for each one. Notice that this doesn't preclude t
he posibility of having a concave subset with a CM that lies outside. But, as we shall see,
as the electrode resolution is finite, the posibility of having highly non compact shapes is unlikely, and they
tend to have a CM inside or near their borders. In the figure \ref{disconnectedsets} B we show this with experimental
data (we are using evoked activity data, as explained in section \ref{sec:evocada}). In graph Bi we
sepparate the CSD data into the sink and sources set. We take sinks as example. In the graph Bii
we label each disjoint component, obtained by a single pass algorhythm. There are 25 components, and
the bigger two are non convex. In the
subfigure Biii we indicate the CM of only those components that are above certain threshold of
integrated intensity (the circle size  is proportional in area to such intensity, there are five CM to be
seen). Each CM gets listed as a probable step in a trajectory.


\begin{figure}
  \includegraphics[width=0.9\textwidth]{Figuras/ExplicaDisconexos01.pdf}
  \caption{ Explanation of the geometrical concepts used on our analysis.
    In the Row A of fiugres
    with an artificial example the concepts of connectedness and convexivity.
    The subfigure Ai shows
    three sets, $A$, $B$, and $C$. $A$ fails to be convex because it is concave,
    and $C$ fails to be
    be convex because it is made of two disconnected components.
    In each of them is possible to draw a line
    connecting two points inside the set that is not entirely inside the set.
    In $B$ that is not possible, the set $B$ is convex in a mathematical sense.
    In subfigure Aii we made some modification to show that
    each set can be made of many disconnected components,
    and each one of them can be convex or not.  In Aiii we show the center of
    mass for $A$ and $B$
    taken as a whole, while for $C$ we break the set into each of the disconnected
    components and calculate each CM separately. There is no guarantee that each
    disconnected component would be convex, but as the resolution of the data is finite,
    small components are more likely to
    be convex. In figure B we show everything with a frame of experimental data.
    In Bi we sepparate the
    data into three sets, Sources, Sinks, and Zeros.
    We discard the Zeros and keep the active ones.
    In Figure Bii we take the Sink Set and separate it into disconnected components,
    they are labeled from one to 25 in this particular example,
    indicated by color
    For each subset we sum over its mass, and 
    considerate for the next step those who are above certain threshold. The
    next step is shown in Biii, where we plot the Center of Mass for each of
    such subsets. Deviations
    from convexity are tolerable,
    as most of the mass is concentrated in the inner electrodes
    of each set.
  }
  \label{disconnectedsets}
\end{figure}



Theoretically, CSD would divide the set of measuring points into three subsets: the set of all sources, the set of all sinks, and the border between them i.e, the set of all measuring points that have a positive CSD  value, the set of those who have a negative CSD value, and the set of those points having exactly the zero value in CSD space.
Due to the finite precision of the measurement,
it is imposible to define a precise zero value.
The measurements produce a function over the grid elements (the electrodes).
Noise between patches corresponding to a single set often makes bridges
between components, making identification of poles inexact.
In order to avoid this we create a ``thick zero set'' of all the recording
points whose CSD values are inside an error interval around zero.
This separates more sharply the components of the set,
represents our ignorance of precise borders and helps
us to simplify the following analysis.

The ``thick zero set'' would help us to sepparate the sinks and sources sets into
its disconnected components, which would correspond then to clear patches of
definite activity. If each one of these patches could be
interpreted as an active unit,
we may asign to it a ``center of activity'' in order to follow their displacements.
A Center of Mass for each instantaneous disconnected sink/source would provide us
a pragmatic measurement of an ``active locus''. We can then calculate at each time
frame the CM for each one of those components before tracking displacements of activity.


In each component, indexed $k$, at a given time $t$,, we have a number of points
(representing electrode centers)
with CSD of the same sign, indexed $j$. Then, the CM of the $k$-th
component would have the usual definition:
\begin{equation}\label{cmparadisj}
   \langle q(t) \rangle_k =\frac{\sum_j q_{j,k} (t) I_{j,k} (q_{j,k},t)}
           {\sum_j I(q_{j,k},t)},
\end{equation}
where $I(q_{j,k},t)$ is the CSD at point $q_{j,k}=(x_{j,k}, y_{j,k})$ at time $t$,
and the
sign of our ``density mass function'' eliminates itself graciously.
These disconnected components, while not rigorously convex,
are not largerly spread over the space of the measurements.



\subsection{CM and trajectories}

We must not take out of sight the phenomena which is producing the CSD distribution:
neurons have a finite size, and the effect of their action a delimited extent.
Every patch that we can identify as a disconnected component of the Sources/Sink
sets indicates a  joint active set of units. A center of mass for each disjoint
component would then indicate an average locus for this active group, a putative center of activity to which we can give a position, intensity and time. An instantaneous snapshot of such information would depict the centers of activity of an ordered structure of neurons and their relative intensities. But it is in the successive depiction of a series of such snapshots that this analysis can show interpretative power. The appearance , displacement, and disappearance of such centers could reveal very subtle details of the dynamics of neural activity. Very small, fast displacements of the Centers of mass, may be occur due to slight asynchronous action between neurons which belong to the same active group.
Larger displacements and  longer intervals (more than two or so firing periods)
could indicate physiological connections between different groups or units.

The next step in our analysis is a procedure to systematically associate to each CM
at time $t$ a ``successor'' at time $t+\Delta t$, where $\Delta t$ is the sampling
time step. Let $q_1=(x_1, y_1)$ be the coordinates of the first point and $q_2=(x_2,y_2)$,
the corresponding coordinates of the second.
Remember that these coordinates can be now
rational numbers
respect to the grid of electrodes,
as CM are not restricted to be  at the grid points.
The distance is:
\begin{equation}
d(q_1,q_2):=\sqrt{(x_1-x_2)^2+(y_1+y_2)^2}.
\end{equation}
The integrated intensity of a CM is simply the denominator in the
eq. \ref{cmparadisj},
that is, the total ``mass'' obtained by integrating the CSD in a specific component:
\begin{equation}
  I_k(t)=\sum_j I (q_{j,k}, t).
\end{equation}
We specify then a tolerance $\delta$. If a certain $t$ a CM is below $\delta$ distance from another at $t+\Delta t$, we consider the latter the successor in time of the former. By continuously applying such procedure, we can begin to trace trajectories of the CMs, as they appear, wander, and fade. 

When a specific trajectory cannot find a successor point that is above the
threshold intensity, we assume that it has ended, so we do not concatenate
points that are separated in time by more than $\Delta t$.
In order to concatenate different
trajectories amore rigorous analysis of their origin and interpretation
should is due and is beyond the scope of this work.



 \subsection{The Procedure}
 
 In order to perform the analysis described here, we summarize the steps.
 \begin{enumerate}
 \item Acquire high density electro-physiological data. The data must be two
   dimensional
   in space. Distance between the recording sites must be of the order of the typical
   size of the neurons involved or less, and such sites must be on a dense enough grid to perform numerical difference operations over them.
   The sampling frequency should be enough to detect sub-threshold activity,
   of at least 3kHz. Also, the data should come from structured tissue.
   Unordered, highly homogeneous tissue, will not yield iterpretable results. 
 \item Calculate the CSD from the data. Any method may be used,
   but computer expensive techniques are discouraged if the
   number of channels is above the thousands.
   Common sense and tests are guidelines here.
   We recommend simpler, difference based approaches wherever possible.
   If many electrodes fail near the r.o.i.,
   kCSD is a good alternative if done with
   parallel routines.
   If too many electrodes are failing over the r.o.i.,
   it is better to discard the data. 
 \item Separate the CSD data into three sets: sources, sinks, and border or indiscernible (below error magnitude) data
   which we call the zeros set. Again, sensible criteria have to be defined.
   A rule of thumb would be to use the mean of the peaks of the noise  of the recording as the amplitude for the ``thick zero'' range of values. 
\item Perform disconnected component analysis in the sources and sinks sets, frame by frame.
  A single pass algorithm here is  adequate \cite{Vincent91, Abubaker07}.
\item  Obtain the CM for each component, using eq. \ref{cmparadisj}. Discard
  all CM that are below some significant threshold value. A rule of thumb is one tenth of the average value for all CM,
  or those who are in the lesser decil. Anything that is around the peak
  values of the noise integrated over around 5 electrodes would have to be discarded.
\item For each frame and its successor, and for each CM, perform a ``most probable successor'' detection. Here we must put numerical criteria according to the data avaible, i.e. how far away and how different in intensity should be allowed a CM to be at time $t+1$ to be considered succesor in time of another at time $t$.
\item ``Connect the dots'', that is, apply succesively the step above for each CM, until it stops having a succesor or its intensity falls below the error criteria. Then plot the succesive CMs as one trayectory.
\item Plot all the trayectories and sort visually those that may have physiological interpretation from those who may be artifacts of the method. This may need some form of expert validation.
 \end{enumerate}
 

 
\section{Examples from Electro physiological data}

The necessity to create such analysis came originally from the study of
electro-physiological data obtained from the rat's hippocampus, using
a very dense MEA, the BioCAM of 3Brain \cite{BioCAM}. 
The highly layered CA structure of the hippocampus provides strong, discernible patterns of Sources and Sinks. Following the pattern of activity using Center of Mass analysis directly from the data, though, proves unfruitful, due to the non-convex nature of the somatic layer. Using CSD and disconnected component tracking over the data provided the right conceptual framework to follow the displacement of active areas during highly active events, such as epileptic-like activity that was pharmacological
or traumatically induced. The proposed analysis would not produce vector averages over different simultaneously active areas. On the contrary, it would separate them and track them independently. So the most violent epileptic burst provided us with the data necessary to prove the robustness of the method.
All the experimental data comes from a series of experiments
performed on slices of the rats hippocampus. 



\subsection{Application to evoked activity in idem.}\label{sec:evocada}

Evoked activity by electrical stimulus yields simpler and cleaner patterns in the LFP
measurements. This facilitates the exemplification of our analysis, so we shall
start the exposition of the usage with these kind of data.
The data was obtained by stimulating certain specific points on the dentate gyrus
of the slice of hippocampus with different voltage intensities. Details of the experiment
have been published elsewhere \cite{Franco2018}. We shall borrow some of
the experimental data for illustrating our purposes.

The LFP is averaged over three events, produced by three succesive
electrical stimuli in the same point at constant voltage. The
response of the cells is allmost identical in each event, which
allows us to use averages as a denoizing mechanism. After averaging
the syncronized data we perform kCSDA, as in this case we end up
with many saturated electrodes and dCDSA is not viable.
In the figure \ref{lfpevocada} we show some frames of the post-stimulus
acticity as a heat map, along
with a diagram of the somatic layers and the
stimulus point. The saturated area is quite large, and covers
a large section of the GD structure. In this case we shall
restrict the CM track analysis to the respondent area in CA3.

Once that we have the CSD representation, we select channels outside
the r.o.i. and use them to obtain the threshold for sepparating
the Sink and Sources sets. Then we procede as stated in the previous section.

The figure \ref{lfpcsdcm}  compares the representation of the activity in
the LFP, CSD and CM spaces. It has been exhaustively proven that the CSD
representation provides better spatial localization of active sites, but the
CM can serve as better dinamical tracker of action. The figure ilustrates
that point by choosing the time interval of response in CA3 after the stimulus.
CM representation permits us to track the activity with less
interference from the saturated region. The small, clearly localized centers of
mass separate clearly from the big patch over the saturated region.
In the figure \ref{evotracks} we concatenate the succesive centers of mass
for both sinks and sources and show how the anterograde reaction is
very well localized over the Pyramidal Stratum at precise moments after
stimulus.

The treatment of the data is shown in figure
\ref{CSD_Evocada01}.  The subfigure A  shows a diagram of the hippocampus, with
the somatic layers (CA and DG) as thick black lines and the stimulus point
shown as a blue arrow. The $a,b,c,d,e$ dots are selected electrodes for
showing LFP and CSD traces. Those are shown in the subfigure B. In subfigure
C the same data is presented as a spatiotemporal color map that
shows the expected poles of firing neurons transverse to the
pyramidal stratum.  In the subfigure D we show the CSD for a series of time frames
over the square delimited in the subfigure A. The idea for treating the different
patches separately becomes evident here. The sources and sinks form patches
with an apparent anterograde movement from CA3 to CA1.
The patches change polarity
as espected, with the somatic layer becoming a strong sink between 3 to 4 ms
after the stimulus, and then, at the 6 ms, inverting its polarity in a way that suggest
successive activation of groups that are represented as movement of the patch.


\begin{figure}[h]
  \includegraphics[width=0.9\textwidth]{Figuras/CSD_EvocadaYTrazos.pdf}
  \caption{ CSDA for evoked activity and visual inspection of disjoint
    components. In A we present a diagram of the hippocampus slice over
    the acquisition device. The somatic layers of CA and DG are represented as
    thick black lines, and the stimulus site is indicated by a blue arrow. A selection
    of six contiguous electrodes labeled $A-F$ in the r.o.i. are presented as blue
    dots. In B we show the LFP traces and their CSD representation for those
    selected electrodes.
    In C we shoe the CSDA for various timestamps in te sub-region
    indicated in A as the r.o.i, which covers a square of 24 by 24 electrodes over CA3.
    in the first image,  4ms after the stimulus,  a
    shaded strip indicating the approximate location of str. pyr. A model neuron is
    drawn for scale. At 4ms after the stimulus, the saturation of the electric shock
    recedes and we can began to appreciate real neuronal activity on the responsive
    sector of CA3. A great sink appears over the somatic layer, surrounded by two
    sources and some smaller patches of activity. At the following frame the patch
    appears to have moved anterogradly, towards CA1. Next frame, the sink appears
    to have inverted its polarity as do their surrounding patches, and the movement
    continues until approximately 9ms after stimulus, where the activity ceases to be
    discernible from noise.}
  \label{CSD_Evocada01}
\end{figure}


 \includegraphics[width=0.9\textwidth]{Figuras/CMVariousTimes01.pdf}
+  \caption{ The CM analysis over the same images as figure \ref{CSD_Evocada01} C. Some trajectories
+    are calculated before the 4ms time stamp. The triangular mark shows the later point of the
+    trajectory, either the end or the last calculated position until that moment. The circular
+    mark is the start of the trajectory. We can see that activity moves form CA3 distal to
+    proximal in alternating waves of action (hiperpolarization, firing and repolarization).
+  }\label{cmevocada}


The evoked activity seems to yield much cleaner results, as we can now compare the trajectories of successive, very similar events. Our CSD and CM analysis can be used to detect consistent features in these experiments, such as the direction the waxing and waning of activity. Small deviations from the trajectory in successive experiments that are above the numerical error could indicate inconsistencies in the network below. 

\subsection{Application to facilitated activity}

%%%% *********** Vas aqui

\subsection{Application to epileptic-like activity in the hippocampus}
Chronologically speaking, we performed first the analysis to the most complicated data (numerically and qualitatively), which where a series of epileptic burst that occurred after 15 minutes of application of 4AP to living slices of hippocampus. The burst occurred with activity spreading over many sites simultaneously. This turned out to be a test of the applicability and robustness of our method. The necessity of having good error value for the zero set was more apparent here than anywhere else. Also the posibilty to create new structures of numerical data was first seen here.  But also the most strickig features of the propagation of activity, coded now as trajectories of the CM, could be seen here, as the epileptic burst covered the whole CA structure. A sample of the raw data in $\mu V$ can be seen in the figure \textbf{Tururu}. Three key moments in the epileptic burst have been selected, one before the onset of the attack, one at the peak and one at the middle of the subsequent waves. Traces from selected electrodes along the line labeled A-F are presented in the subfigure D. 

In figure \textbf{Tururu2} we present the CSD for the same selected moments.

In the figure \textbf{Tururu3} we present our method using the peak moment as an example. 

\subsection{Application to spontaneous activity in idem.}

We had one experiment in which we could trace very nicely the data on DG and CA simultaneously without adding excitatory drugs. 

\subsection{Counterexample: the stratial neuronal tissue}

bla bla.

\bibliographystyle{abstract}
\bibliography{../Reportes/BiblioReportes01} 


\end{document}


