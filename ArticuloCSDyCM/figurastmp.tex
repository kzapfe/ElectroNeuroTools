%% 1

\begin{figure}[h!]
  \captionsetup{singlelinecheck=off} 
  \includegraphics[width=0.9\textwidth]{Figuras/explicadisconected.pdf}
  \caption[test]{

      
    Explanation of the geometrical concepts used on our analysis.
    In the Row \textbf{A} of figures we show
    with an artificial example the concepts of connectedness and convexity.
    The subfigure \textbf{A\ts{1}} shows
    three sets, $A$, $B$, and $C$. $A$ fails to be convex because
    it is convex,
    and $C$ fails to be
    be convex because it is made of two disconnected components.
    In each of them is possible to draw a line
    connecting two points inside the set that is not entirely inside the set.
    In $B$ that is not possible, the set $B$ is convex in a mathematical sense.
     In \textbf{A\ts{2}} we show the center of
    mass for $A$ and $B$
    taken as a whole, while for $C$ we break the set into each of the disconnected
    components and calculate each CM separately. There is no guarantee that each
    disconnected component would be convex, but as the resolution of the data is finite,
    small components are more likely to
    be convex. In figure \textbf{B} we show the procedure
    with a frame of experimental data.
    In \textbf{B\ts{1}} we separate the
    data into three sets, Sources, Sinks, and Zeros.
    We discard the Zeros and keep the active ones.
    In Figure \textbf{B\ts{2}}
    we take the Sink Set and separate it into disconnected components,
    they are labeled from one to 25 in this particular example,
    indicated by color
    For each subset we sum over its mass, and 
    considerate for the next step those who are above certain threshold. The
    next step is shown in \textbf{B\ts{3}},
    where we plot the Center of Mass for each of
    such subsets. Deviations from convexity are tolerable,
    as most of the mass is concentrated in the inner electrodes
    of each set.
  }
  \label{disconnectedsets}
\end{figure}

%% 2

%\begin{figure}[h!]
%  \captionsetup{singlelinecheck=off} 
%  \includegraphics[width=0.9\textwidth]{Figuras/lfpevocado01.pdf}
%  \caption[test]{
%    A set of frames showing the stimulus and posterior activity.
%    The frame titled ``0 ms'' is just before the stimulus was
%    made.  A diagram of the Granular and Pyramidal Layers is shown
%    against the noise background. The black rectangle delimits the
%    r.o.i. and the black arrow shows the site of stimulation in GD.
%  }\label{lfpevocada}
% \end{figure}

%% Era la  3, pero ahora es la dos

\begin{figure}[h!]
   \centering
   \includegraphics[width=0.9\textwidth]{Figuras/figure03.pdf}
   \caption{
      Electrode signal across some selected electrodes over the CA3 region.     
      In subfigure (A)
      the raw traces of the selected electrodes are shown, and in (B) the
      csd traces come along for comparison. In this last subfigure the
      vertical scale is in arbitrary units.
      In subfigure (C)  the GD and CA3 structure are revealed
      by the LFP measurement 8.5 ms after the stimulis,
      their general shape recognizable.The selected electrodes are indicated
      by black dots and the square encompasses the r.o.i. 
   }
  \label{trazselectevo}
\end{figure}


%% 4

\begin{figure}[h!]
  \captionsetup{singlelinecheck=off} 
  \includegraphics[width=0.95\textwidth]{Figuras/lfpcsdcm_completa_evocada3R7.pdf}
\caption[tu abueala]{
  Comparison of the activity represented as LFP, CSD and CM.
  The time after
  stimulation in ms is shown in the central column for each row.
  In this series of frames a relatively big patch of the electrodes
  still appear saturated, but even so the CSD shows good localization
  of the relevant deflections from the neutral or noise base signal.
  The CM representation concentrates the locus of action even more.
  The circular dots in the third column represent the center of mass
  of coherent active subsets.
}\label{lfpcsdcm}
\end{figure}


%% 5

\begin{figure}[h!]
   \captionsetup{singlelinecheck=off} 
   \includegraphics[width=0.95\textwidth]{Figuras/lfpcsdcm_sub_evocada3R7.pdf} 
   \caption[test]{
     The same as previous figure, but focusing on the
  region of interest, including the stratum pyramidale of CA3. A different
  color scheme was chosen on the last column, so as to provide borders
  that show how the disconnected components where chosen. 
}\label{lfpcsdcmsub}
\end{figure}


%% 6, bueno, ahora es la 5!

\begin{figure}[h!]
   \captionsetup{singlelinecheck=off} 
 \includegraphics[width=0.95\textwidth]{./Figuras/trayevocadasub.pdf}
 \caption[test]{
   The evoked trajectories in the r.o.i. We begin to track them
    at the 2.6 ms mark, well after letting the saturation of the stimulus recede.
    At 3ms a very clear sink trajectory is moving retrogradly towards CA2,
    shepherd by two shorter source trajectories. At 5.6 a depolarization
    trajectory starts, following almost the same path and presenting also its
    two companions, ending around the 7.7ms timestamp. No significant
    trajectories are observed after the 8.5ms mark. The triangular
    marks indicate the last active point added to a trayectory, indicating
    roughly the direction of movement.
 }\label{evotracks}
\end{figure}


%% 7

\begin{figure}[h!]
   \centering
   \includegraphics[width=0.9\textwidth]{Figuras/figure07.pdf}
    \caption{
      A selection of traces of the recordings for various electrodes.
      In the subfigure A, the selected electrodes
      are shown against the standard deviation of the recording of the
      potential, which gives a good indication of the more active
      sites, and therefore, of the structure of the hippocampus.
      The electrode labeled ``a'' is over CA3 proximal, on the other
      extreme ``h'' is in the middle of CA1.
    }\label{trazfaci}
\end{figure}


%% 8

\begin{figure}[h!]
   \captionsetup{singlelinecheck=off} 
   \includegraphics[width=0.95\textwidth]{Figuras/tablalfpcsdcm-facilitada.pdf}
   % falta svg TOTAL
   % Se la estas haciendo de emocion a Rafael con Esta.
   % Dejala fermentar.
\caption[test]{
  Comparison of the activity represented as LFP, CSD and CM,
  in the same style as fig. \ref{lfpcsdcm}, but for facilitated activity.
  In this case the time counter starts at some moment around 80 ms. before the
  onset of the burst of activity.
 \komment{Observa como cambié los tiempos. Como tu sugerias quedaba raro, dado que
    que la actividad comienza antes del milisegundo 90, asi que tendríamos
    mucha actividad en tiempos negativos }.
}\label{lfpcsdcmfaci}.
\end{figure}

%% 9 ahora 8

\begin{figure}[h!]
  \captionsetup{singlelinecheck=off}
  \includegraphics[width=0.95\textwidth]{./Figuras/figure09.pdf}
\caption[test]{
 (A)The trajectories for  the facilitated activity, in
  sequence of apparition around the burst of action.
  The blue line indicates the location of the str. pyr. and the
  orange one the location of the Dentate Gyrus. The zero has been chosen
  as in figure \ref{lfpcsdcmfaci}. Sink trayectories are shown
  as light blue, and Source as red. 
  \\ (B) All the trajectories shown together.
  \romment{¿En este cuadro se van añadiendo distancias?} \\
  \komment{Pues si, es decir, se va viendo el desplazamiento.} \\
  \romment{¿Van apareciendo nuevos origienes?} \\
  \komment{Si, también}
  \romment{ Hacer una figura siguendo UNA trayectoria de largo alcance.}
  \komment{ Mientras no tengamos un criterio para ``enlazar'' trayectorias
    más alla de la cercanía (que es el que uso), no tenemos trayectorias de
    largo alcance. Aunque hay algunas de alrededor de unas 150 o 200 micras.}
}\label{facitracks}

\end{figure}


%% ahora esta es la 9.

\begin{figure}[h!]
   \centering
   \includegraphics[width=0.9\textwidth]{Figuras/figure10.pdf}
    \caption{
      LFP standard deviation and traces of the recordings for various electrodes.
      In the subfigure (A), the selected electrodes
      are shown against the standard deviation of the recording of the
      potential. This is an indicator of the more active
      sites, and therefore, of the structure of the stratum pyramidale.
      The electrode labeled ``a'' is putatively over GD, while ``h''
      is in the medial CA1.}\label{traz4ap}
\end{figure}



%% 10

\begin{figure}[h!]
   \captionsetup{singlelinecheck=off} 
  \includegraphics[width=0.9\textwidth]{./Figuras/LFPvsCSD_4AP.pdf}
  \caption[test]{
    Comparison of LFP and CSD representations
    at the moments of the epileptic burst. In the first row
    we present the data as measured, no denoising or filtering has
    been applied. The units of measurement are in $\muV{}$. In
    the middle row a direct application of a difference operator
    gives us the CSD. This sort of operators are very sensitive
    to noise, in this case, spatial hard edges, which become
    greatly exaggerated. In the third row we present the same
    data after a Gaussian spatial filter ( Gaussian Blur) has
    been applied. }\label{lfpycsd4ap}
\end{figure}

%% 11

%% 12

\begin{figure}[h!]
   \captionsetup{singlelinecheck=off} 
  \includegraphics[width=0.9\textwidth]{./Figuras/Fronteras2sigma.pdf}
  \caption[test]
    {The detection of disconnected components for the Sink and Source
    Sets. We take the the threshold value as two times the
    standard deviation of the whole data in CSD representation. In
    the upper row the data without the Gaussian Blur filtering is shown.
    Although the location of some of the active sets is clear, a lot of the
    joint activity gets hidden under the spatial noise. In the lower
    row we repeat the detection with the smoothed data. Large, coherent
    patches of activity are clearly shown, and the structure of the
    stratum pyramidale is apparent. 
  }\label{csdfrontera}
\end{figure}


%% 13

\begin{figure}[h!]
   \captionsetup{singlelinecheck=off} 
  \includegraphics[width=0.95\textwidth]{./Figuras/SeqTray4AP01.pdf}
  \caption[test]{
    The activity in the Center of Mass and Trajectories representation.
    A single epileptic attack is shown from slightly before the onset until the
    last wave of the attack. Sink and Source trajectories are shown in blue
    and red, respectively. In order to not saturate the latter subfigures
    we take a persistence value of 2 ms, meaning that we do not shown
    trajectories 2 ms after they have ended.}\label{tray4ap}.
 \end{figure}


%% 14

\begin{figure}[h!]
  \centering
 \captionsetup{singlelinecheck=off} 
  \includegraphics[width=0.9\textwidth]{Figuras/figure14.pdf}
  \caption[test]{
    (A)LFP stratial Tissue, some samples. There are no discernible
    structures. The apparent active set at the lower left corner
    is an artifact due to some bubbles in the experiment.
    (B) After 30 seconds of recording, no visible trajectories can be found. The minimum displacement criteria is not full-filed for most of the putative centers of mass.}
  \label{cuadestriado}
\end{figure}



