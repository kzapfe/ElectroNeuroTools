CSDA

 
As it is mentioned in the section ``data analysis''  above, the first step is to detect the electrodes that are failing
because of saturation or lack of detection. The criterion is that a electrode
which shows less standard deviation than the one expected by the noise, then
the data from that channel is discarded. In this section our threshold for
discarding them was if the standard deviation was less than one half of the
average from the standard deviations of the background noise.
Every channel that was discarded was replaced by the instantaneous average
of its 8 neighbours (discarding some of them if they were also saturated).

In order to obtain the CSD from the LFP we first had to remove the excessive
hard edges we smoothed each frame with a Gaussian Filter with a half-width of
42 micrometers (a center to center interelectrode distance). At this moment
the numeric data was ready for the CSDA. We follow the usual ohmic, isotropic,
and homogeneous assumption about the extracellular medium, so that
the CSD is expressed by the following relation:

**** Poner ecuación de CSD-definition aquí *****,

where I_m is the CSD, \sigma is the conductivity tensor, and V is the
LFP.

Due to the high spatial resolution of the BrainWave BioCam, we can perform a very
precise CSDA with a finite difference method. The best suited tool for this
is the following numerical Laplacian operator:

*****Poner la ecuación del operador Laplaciano-Lindberg aquí. *****

In two dimensional rectangular grid data,
this operator is as close as possible to a rotational invariant operator
(Lindberg, 1990). This reduces as much as possible the effects of the orientation
of the grid of electrodes in respect to the orientation of the pyramidal
neurons, providing us with more sensitive patches of sinks and sources,
and reducing the effect of ``crosses'' that occurs with non-convex
operators. The procedure is to apply this operator
 as convolution to each frame (each time step) of the data. To
deal with the border, we padded the data with a duplicate of the first
and last rows and columns and then discarded the resulting channels
after the convolution was made. 


CMA (Center of Mass Analysis)

At each time frame, the data in the CSD representation can be classified in 
three different sets: Sinks, Sources, and Neutral. Due to the presence of
noise, the Neutral set has to be defined on empirical grounds. We calculated
the average extrema of the noise over the channels, and set the larger of
them as our threshold value for the neutral set. This results in a band
that separates the Sinks and the Sources set by a thicker neutral set.

These two active sets (the sinks and sources) are made of disjoint patches, that
is, the sets are disjoint. We used a single pass algorithm
(Vincent and Soille, 1991) to detect each connected component from each set. This gives
us a formal labeling of each patch at each instant of time. Each such component is
connected in itself but disconnected to the other components of the corresponding set.
So each component lends itself to the obtaintion of its center,
a formal point to which we can
assign the action averaged over space, without mixing information from
distant units. We use the common definition of ``Center of Mass'', i.e. a
vector average of the instantaneous CSD over all the channels that
form a patch:

*** poner ecuación de Centro de Masa aqui ****,

where the angle brackets denote averaging over the connected component, m
is the absolute instantaneous value of the CSD at each channel, and the x_{k,j} are the coordinates
of the channel. The index j runs through the elements of the patch.
Our surrogate data now are the corresponding ``centers'' and
their total ``signed mass''
(the sum of all the contributing channels CSD to each center).



Trajectories

To obtain the trajectories we start with the Center of Mass Data at
the first  time frame. We perform the following
steps for sinks and sources separately. We  compare the locations of all
the Centers at one frame with those of the following frame, and we select the closer to
each one as its continuation, if its total density is above a certain
threshold, and if not, we discard it. Proceeding in such a manner, frame
by frame, we can obtain the putative trajectories of action of the
relevant patches. A trajectory ends when there are no centers closer
to the last calculated center point, or are to low in intensity to consider.
For the animation, we selected only the trajectories that had more
than 5 steps of existence and their average value was above certain
threshold. This was done in order to not saturate the image with too
many small trajectories. Analysis of the whole data and formal details
will be presented in a following paper.



