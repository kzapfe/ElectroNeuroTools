\documentclass[serif]{beamer}

%\usepackage[numberedbib]{apacite}
\usepackage[utf8]{inputenc}
\usepackage[spanish]{babel}
\usepackage{natbib}
\usepackage{bibentry}
\bibliographystyle{apalike}


\begin{document}

\begin{frame}
    \frametitle{The track of activity and MEAs}
    Given a record of LFP is it possible to discover
    some track that the neural activity traces in a meaningful
    way?
    \begin{center}   
    \includegraphics[width=0.55\textwidth]{ImagenManjarrez01.jpg}
    \end{center}
    
    ZC Chao et all, \emph{Effects of Random External Background Stimulation
      on Network Synaptic Stability After Tetanization}, Neuroinformatics, 3, 2007.\\

    Manjarrez et all, \emph{Computing the Center of mass for traveling alpha waves in
  the human brain}, Brain Research 1145, 2007.   
  \end{frame}


   \begin{frame}
     \frametitle{CM is good, LFP a mass is not}
     \begin{equation}
       \langle x_k(t) \rangle =\frac{\sum_j x_{j,k} (t) m_{j,k} (t)} {\sum_j m_{j,k}(t)}
     \end{equation}   
     \begin{itemize}
       \item LFP (as most Potential measures) has no defined zero.
       \item The Spike Density is a true density, but has an heuristic parameter.
     \end{itemize}
    
     
     \begin{tabular}{ c c }
      \includegraphics[width=0.4\textwidth]{ComoManjarrezHaceCM01.png} &
       \includegraphics[width=0.4\textwidth]{ComoChaoHaceCM01.png} 
     \end{tabular}

    \end{frame}
  

  \begin{frame}
    \frametitle{A Natural density is at hand}
    Assuming constant, isotropic propiertes of the extracellular medium:
      \begin{equation}
       I_m/\sigma=-\nabla^2 V
      \end{equation}
      \begin{equation}
       \frac{\partial \rho}{\partial t} + \nabla \cdot \mathbf{J}=0 
      \end{equation}
      What about the sign?
      The sign help us to recognize two disjoint sets of activity:
      the Sources, and the Sinks.
      We have two different non overlapping measures: the
      Sink Density and the Source Density.
  \end{frame}
     
     

  \begin{frame}
     \frametitle{Implementation: naïve search}
     A direct approach proves useless, as the activity occurs at separated
     sites along CA1 and CA3. The concave geometry of our slice
     produces a centre of mass outside the CA, and without physiological
     interpretation.

       \begin{tabular}{ c c }
      \includegraphics[width=0.5\textwidth]{CSDSeminarioTraub.png} &
       \includegraphics[width=0.35\textwidth]{DiagCSDTraub.pdf} 
     \end{tabular}
     
  \end{frame}

    
  
  \begin{frame}
     \frametitle{A Natural Decomposition}
     Both Sink and Sources Set form patches over the interest region.
     There is a noice level that forms a ``thick'' zero. This allows us
     to sepparate each set into Disconnected components.
     \begin{tabular}{ l r }
      \includegraphics[width=0.35\textwidth]{DisjuntosUmbral.png} &
       \includegraphics[width=0.5\textwidth]{CSDCMSemTraub.png} 
     \end{tabular}
  \end{frame}

  \begin{frame}
    \frametitle{CM dynamics: putative trajectories for activity}
    Joining CM to CM at each time frame with a proximity threshold gives us
    putative trajectories for the activity, allready separated by type (either
    relasing or absorving positive ions of the enviroment) and
    site.
    \begin{center}
      \includegraphics[width=0.55\textwidth]{TrayectoriasLimpiasNumeros01.png} 
    \end{center}
  \end{frame}

  \begin{frame}
    There is still work to do here, and the Devil hides in the details.
    \begin{center}
      \includegraphics[width=0.55\textwidth]{TrayLimpSemTraub.pdf}
      \end{center}
  \end{frame}


  \begin{frame}
    \frametitle{And now for something completely different \ldots}
    There is other density avaible directly from the data:
    Power density.\\
    It also admits a decomposition into different interpretable sets:
    a decomposition by frequency bands.
    \begin{center}
      \includegraphics[width=0.55\textwidth]{DisconnectedByBands01.png}
      \end{center}
  \end{frame}
  

 
%  \begin{frame}{Bibliography}
   % \bibliographystyle{apacite}
  %  \bibliography{BiblioReportes01}
 % \end{frame}
\end{document}

