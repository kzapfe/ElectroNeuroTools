\documentclass[10pt, serif]{beamer}

%\usepackage[numberedbib]{apacite}
\usepackage[utf8]{inputenc}
\usepackage[spanish]{babel}
\usepackage{natbib}
\usepackage{bibentry}
\usepackage{color}
\usepackage{beamerthemeshadow}

\usetheme{Berlin}
\usecolortheme{spruce}


%bibliographystyle{apalike}


\begin{document}

\begin{frame}
    \frametitle{The track of activity and MEAs}
    Given a record of LFP is it possible to discover
    some track that the neural activity traces in a meaningful
    way?
    \begin{center}   
    \includegraphics[width=0.55\textwidth]{ImagenManjarrez01.jpg}
    \end{center}
    
    ZC Chao et all, \emph{Effects of Random External Background Stimulation
      on Network Synaptic Stability After Tetanization}, Neuroinformatics, 3, 2007.\\

    Manjarrez et all, \emph{Computing the Center of mass for traveling alpha waves in
  the human brain}, Brain Research 1145, 2007.   
  \end{frame}


   \begin{frame}
     \frametitle{CMA may work, but an adequate measure density is needed.}
     For a series of sets, indexed by $k$, we want to find their ``center of
     mass'', that is, their vector average, at the moment $t$. let $j$ run over
     the elements of the set, and $x$ be their coordinates, then the $k$-th
     center of mass at time $t$ is given by:
     
     \begin{equation}
       \langle x_k(t) \rangle =\frac{\sum_j x_{j,k} (t) m_{j,k} (t)} {\sum_j m_{j,k}(t)}
     \end{equation}

     Then, $m_{j,k}$ has to be a measure density function, in more or less the usual
     definition (positive and with finite integral over the sets).
   \end{frame}

   \begin{frame}
     \frametitle{Manjarrez choice}

     Manjarrez et all. use the positive part of the signal, with respect to
     the measurements device zero. As the signal is a potential, this
     is very loosely defined. Even more, it disregards polarization events.
     At a very rough scale may give adequate results.
          
     \begin{itemize}
       \item LFP (as most Potential measures) has no defined zero.
       \item Disregards half of the activity that appears in the recording as
         ``negative'' potentials.
       \item Large scale use.
     \end{itemize}

     \begin{tabular}{ c c }
      \includegraphics[width=0.36\textwidth]{ComoManjarrezHaceCM01.png} &
       \includegraphics[width=0.6\textwidth]{ImagenManjarrez01.jpg}
     \end{tabular}

    \end{frame}


   \begin{frame}
     \frametitle{Chao's choice}

     Chao et all. use the firing rate density as density measure. 
     This is a more rigorous choice, and more adequate to
     work with microelectrode arrays, closer to the cellar scale.
          
     \begin{itemize}
       \item Positive function.
       \item Requires external parameter (bin size), can be adjusted on
         heuristics. 
       \item It specializes on one type of action, namely, spikes. 
     \end{itemize}

     \begin{tabular}{ c c }
      \includegraphics[width=0.3\textwidth]{ComoChaoHaceCM01.png} &
       \includegraphics[width=0.5\textwidth]{ChaoFRCAT.png} 
     \end{tabular}

    \end{frame}

   

  \begin{frame}
    \frametitle{A Natural density is at hand}
    Assuming constant, isotropic properties of the extracellular medium:
      \begin{equation}
       I_m/\sigma=-\nabla^2 V
      \end{equation}
      \begin{equation}
       \frac{\partial \rho}{\partial t} + \nabla \cdot \mathbf{J}=0 
      \end{equation}
      What about the sign?
      The sign help us to recognize two disjoint sets of activity:
      the Sources, and the Sinks.
      We have two different non overlapping measures: the
      Sink Density and the Source Density.\\

      \begin{tabular}{ c c }
        \includegraphics[width=0.49\textwidth]{VectorFields.png} &
        \includegraphics[width=0.49\textwidth]{CSDSeminarioTraub.png} 
      \end{tabular}
          
  \end{frame}
     
     

  \begin{frame}
     \frametitle{Implementation: naïve search}
     A direct approach proves useless, as the activity occurs at separated
     sites along CA1 and CA3. The concave geometry of our slice
     produces a center of mass outside the CA, and without physiological
     interpretation.

       \begin{tabular}{ c c }
      \includegraphics[width=0.5\textwidth]{CSD_SemTraub02.png} &
       \includegraphics[width=0.35\textwidth]{DiagCSDTraub.pdf} 
     \end{tabular}
     
  \end{frame}

    
  
  \begin{frame}
     \frametitle{A Natural Decomposition}
     Both Sink and Sources Set form patches over the interest region.
     There is a noise level that forms a ``thick'' zero. This allows us
     to sepparate each set into Disconnected components.
     \begin{tabular}{ l r }
      \includegraphics[width=0.35\textwidth]{DisjuntosUmbral.png} &
       \includegraphics[width=0.5\textwidth]{CSDCMSemTraub.png} 
     \end{tabular}
  \end{frame}

  \begin{frame}
    \frametitle{CM dynamics: putative trajectories for activity}
    Joining CM to CM at each time frame with a proximity threshold gives us
    putative trajectories for the activity, already separated by type
    and location.
    \begin{center}
      \includegraphics[width=0.55\textwidth]{TrayectoriasLimpiasNumeros01.png} 
    \end{center}
  \end{frame}

  \begin{frame}
    \frametitle{Classification of Trajectories}
       \begin{center}
      \includegraphics[width=0.55\textwidth]{PathsComparaTraub.png} 
    \end{center}
 
    
  \end{frame}
  
  
  
  \begin{frame}
    There is still work to do here, and the Devil hides in the details.
    \begin{center}
      \includegraphics[width=0.55\textwidth]{TrayLimpSemTraub.pdf}
      \end{center}
  \end{frame}


  \begin{frame}
    \frametitle{And now for something completely different \ldots}
    There is other density available directly from the data:
    Power density.\\
    It also admits a decomposition into different interpretable sets:
    a decomposition by frequency bands.
    \begin{center}
      \includegraphics[width=0.55\textwidth]{DisconnectedByBands01.png}
      \end{center}
  \end{frame}
  

 
%  \begin{frame}{Bibliography}
   % \bibliographystyle{apacite}
  %  \bibliography{BiblioReportes01}
 % \end{frame}
\end{document}

