\documentclass{article}

%usepackage[utf8]{inputenc}
 \usepackage{fontspec}
\usepackage{amsmath}
%\usepackage[spanish]{babel}
\usepackage{graphicx}

\usepackage{caption}
\usepackage{subcaption}



\setmainfont[Ligatures=TeX]{Oldstyle HPLHS}

\newcommand{\Jd}{\mathbf{J}}
\newcommand{\EF}{\mathbf{E}}
\newcommand{\cond}{\boldsymbol{\sigma}}
\DeclareMathOperator{\diver}{div}
\DeclareMathOperator{\grad}{grad}

\title{CSDA in 2D slices of the hippocampus \emph{done right}}
\author{W. P. K. Zapfe, Whoever Helps}

\begin{document}

\maketitle

\begin{abstract}
As high density electrode arrays become cheaper and more common, 
electrophysiological measurements of activity in neural tissue approach
the resolution of the studied structures. The electrodes measure
directly the Local Field Potential, but it is aknoledged that the
Current Source Density has more resolving power, as it points
more accurately to the actual sources and sinks of ionic current,
and thus, to the more active places in the tissue. Nevertheless,
the role of the conductivity tensor is rarely taken into account
in the derivation of the CSD, which could potentialy lead to
spurious distributions of the sources and sinks and deformation
of their structure, obfuscating the real activity underneath.
Here we show a possible qualitative treatment of the conductivity
across the stratum piramidale of the hippocampus, an expemplification
with two toy models and a possible numerical implementation with
high density real data.
\end{abstract}


\section{Introduction}

As Bédard and Destexhe pointed out in \cite{Bedard11}, the CSDA, as it
is performed usually, rests on
assumptions which could leave important features out. One of these 
assumptions is that the extracellular medium has very homogeneous 
electrical propierties, but this has been shown to not be
the case for dense neuronal layers such as the piramidal
strata in the hippocampus \cite{Holsheimer87, Lopez01, MarioPersonal}.
The evidence mostly points out that resistivity grows 
considerably around the somatic layers. Here we shall investigate
posible influence of this change of resistivity on current density
($\Jd$) and current source density ($I_m$). 



\section{Analytical  derivation of the CSD}

For a carefull and complete derivation and discution of the CSD, 
see \cite{Bedard11}. We start with the charge conservation equation,
\begin{equation}
\nabla \cdot \Jd + \frac{d \rho}{d t} =0.
\end{equation}


\end{document}