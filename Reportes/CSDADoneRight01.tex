\documentclass{article}

%\usepackage[utf8]{inputenc}
\usepackage{fontspec}
\usepackage{amsmath}
\usepackage[spanish]{babel}
\usepackage{graphicx}

\usepackage{caption}
\usepackage{subcaption}



\setmainfont[Ligatures=TeX]{Oldstyle HPLHS}

\newcommand{\Jd}{\mathbf{J}}
\newcommand{\EF}{\mathbf{E}}
\newcommand{\cond}{\boldsymbol{\sigma}}
\DeclareMathOperator{\diver}{div}
\DeclareMathOperator{\grad}{grad}

\title{CSDA in 2D slices of the hippocampus \emph{done right}}
\author{W. P. K. Zapfe, Whoever Helps}

\begin{document}

\maketitle

\begin{abstract}
As high density electrode arrays become cheaper and more common, 
electrophysiological measurements of activity in neural tissue approach
the resolution of the studied structures. The electrodes measure
directly the Local Field Potential, but it is aknowledged that the
Current Source Density has more resolving power, as it points
more accurately to the actual sources and sinks themselves,
and thus, to the more active places in the tissue. Nevertheless,
the role of the conductivity tensor is rarely taken into account
in the derivation of the CSD, which could potentialy lead to
spurious distributions of the sources and sinks and deformation
of their structure, obfuscating the precise place of the
poles underneath.
Here we show a possible qualitative treatment of the conductivity
across the stratum piramidale of the hippocampus, an examplification
with two toy models and a possible numerical implementation with
high density experimental data.
\end{abstract}


\section{Introduction}

As Bédard and Destexhe pointed out in \cite{Bedard11}, the CSDA, as it
is performed usually, rests on
assumptions which could leave important features out. One of these 
assumptions is that the extracellular medium has very homogeneous 
electrical propierties, but this has been shown to not be
the case for dense neuronal layers such as the piramidal
strata in the hippocampus \cite{Holsheimer87, Lopez01, TrevinoPersonal}.
The evidence mostly points out that resistivity grows 
considerably around thick somatic layers, or conversely, that
the conductivity drops considerably there. In ohmic (linear) media,
we expect the following relation to hold:
\begin{equation}
\Jd=\cond \EF,
\end{equation}
that is, the current density $\Jd$ is the result of the
local electric field $\EF$ and the local conductivity, $\cond$.
We usually take that last quantity as an homogeneous, isotropic,
factor, at least over the scales that we are interested in.
The conductivity in inhomogeneous media can be rarely
simplified to a constant, as it can depend both from the direction
of the fields and currents, and vary from point to point. 
Then the above expression is just shorthand notation for the
explicit  \emph{tensorial} relationship:
\begin{equation}
  \begin{pmatrix}
    J_x \\
    J_y \\
    J_z
  \end{pmatrix}
  =
   \begin{pmatrix}
     \sigma_{xx}E_x+ \sigma_{xy}E_y+\sigma_{xz}E_z \\
     \sigma_{xy}E_x+ \sigma_{yy}E_y+\sigma_{yz}E_z \\
     \sigma_{xz}E_x+ \sigma_{yz}E_y+\sigma_{zz}E_z \\
  \end{pmatrix},
\end{equation}
where we allready used the fact that $\cond$ is simmetric when
expressed on cartessian coordinates.
We assume that the electric field can be obtained as the
gradient of the LFP, $\phi$:
\begin{equation}
  -\grad \phi=\EF
\end{equation}
This LFP is the quantity that the microelectrode arrays (MEAs) permits us
to measure on the extracelular medium.
To obtain the sources and sink densities, we use Gauss Divergence Theorem,
stated in differencial form:
\begin{equation}
\nabla \cdot \Jd =I_m
\end{equation}
We perform the divergence operation on the current density, not
on the electric field. Only when the medium is isotropic and homogeneous
we can be sure that both operations coincide within a positive factor.

Here we shall investigate
posible influence of changing conductivity on current density
($\Jd$) and current source density ($I_m$). 


\section{Analytical  derivation of the CSD}


In experiments of this sort performed on hippocampus slices, the
MEAs permit us to measure a sample of LFP over discrete points on a
two dimensional lattice, so we do not have possibility of
calculating currents or electric fields on the third dimension of space.
The slices of the hippocampus are very thin, and we do not consider
the posibility of current going outside the neural tissue sample (in one
direction there is air and on the other the MEA), so we
can reduce our formalism to two dimensional space. We then take
$J_z=0$ and use $(x,y)$ as a cartesian coordinate system aligned
with the MEA. Then the \emph{flat} current density  is:
\begin{equation}
  \begin{pmatrix}
    J_x\\
    J_y
  \end{pmatrix}  =
  \begin{pmatrix}
    \sigma_{xx}E_x+\sigma_{xy}E_y \\
    \sigma_{xy}E_x+\sigma_{yy}E_y 
  \end{pmatrix}=
  \begin{pmatrix}
    -\sigma_{xx}\partial_x \phi -\sigma_{xy}\partial_y \phi \\
    -\sigma_{xy}\partial_x \phi -\sigma_{yy}\partial_y \phi 
  \end{pmatrix}.
\end{equation}
The divergence, in cartesian coordinates, yields:
\begin{equation}\label{csd2dexp}
  I_m=\partial_x J_x+\partial_y J_y=
  -\partial_x( \sigma_{xx}\partial_x \phi +\sigma_{xy}\partial_y \phi)
  -\partial_y( \sigma_{xy}\partial_x \phi +\sigma_{yy}\partial_y \phi). 
\end{equation}
Each one of the terms of the right hand side of eq. \ref{csd2dexp}
has the form
\begin{multline}
  \partial_j (\sigma_{jj}\partial_j\phi) +\partial_j (\sigma_{jk}\partial_k \phi)
  =  \\
  (\partial_j \sigma_{jj})(\partial_j\phi) +\sigma_{jj}\partial_j^2\phi +
  (\partial_j \sigma_{jk})(\partial_k \phi)+\sigma _{jk}\partial_j\partial_k \phi.
\end{multline}
The complete expression is cumbersome to write, but it is important to
have \emph{full} expansion in order to make the numerical considerations
in what follows.
\begin{equation}
  \begin{split}
  -I_m= & (\partial_x \sigma_{xx}) (\partial_x \phi) +
  (\partial_y \sigma_{yy}) (\partial_y \phi)\\
& +(\partial_x \sigma_{xy}) (\partial_y \phi) +(\partial_y \sigma_{xy}) (\partial_x \phi)\\
& + \sigma_{xx} \partial_x^2 \phi + \sigma_{yy} \partial_y^2 \phi\\
& + 2\sigma_{xy} \partial_x\partial_y \phi. 
\end{split}
 \end{equation}
  




For a carefull and complete derivation and discution of the CSD, 
see \cite{Bedard11}. We start with the charge conservation equation,
\begin{equation}
\nabla \cdot \Jd + \frac{d \rho}{d t} =0.
\end{equation}


\bibliographystyle{plain}
\bibliography{BiblioReportes01}



\end{document}
