\documentclass{article}

%\usepackage[utf8]{inputenc}
\usepackage{fontspec}
\usepackage{amsmath}

\usepackage{graphicx}

\usepackage{caption}
\usepackage{subcaption}



\setmainfont[Ligatures=TeX]{Oldstyle HPLHS}

\newcommand{\Jd}{\mathbf{J}}
\newcommand{\EF}{\mathbf{E}}
\newcommand{\cond}{\boldsymbol{\sigma}}
\DeclareMathOperator{\diver}{div}
\DeclareMathOperator{\grad}{grad}

\title{CSDA in 2D slices of the hippocampus \emph{done right}}
\author{W. P. K. Zapfe, Whoever Helps}

\begin{document}

\maketitle

\begin{abstract}
As high density electrode arrays become cheaper and more common, 
electrophysiological measurements of activity in neural tissue approach
the scale of the underlying structures. The electrodes measure
directly the Local Field Potential, but it is aknowledged that the
Current Source Density has more resolving power, as it points
more accurately to the actual sources and sinks. Nevertheless,
the role of the conductivity tensor is rarely taken into account
in the derivation of the CSD, which could potentialy lead to
spurious distributions of the sources and sinks and deformation
of their structure, obfuscating the precise locus of the
poles underneath. In dense neural tissue, such as the
stratum piramidale layer of the hippocampus, clear
deviations from average conductivity have been detected. 
Here we show a possible qualitative treatment of the conductivity
across the layers of the hippocampus, an examplification
with two toy models and a numerical implementation with
high density experimental data.
\end{abstract}


\section{Introduction}

As Bédard and Destexhe pointed out in \cite{Bedard11}, the CSDA, as it
is performed usually, rests on
assumptions which could leave important features out. One of these 
assumptions is that the extracellular medium has very homogeneous 
electrical propierties. This has been shown to not be
the case for dense neuronal layers such as the piramidal
strata in the hippocampus \cite{Holsheimer87, Lopez01, TrevinoPersonal}.
The evidence mostly points out that resistivity grows 
considerably around thick somatic layers, or conversely, that
the conductivity drops considerably there. In ohmic (linear) media,
we expect the following relation to hold:
\begin{equation}
\Jd=\cond \EF,
\end{equation}
that is, the current density $\Jd$ is the result of the
local electric field $\EF$ and the local conductivity, $\cond$.
We usually take that last quantity as an homogeneous, isotropic,
factor, at least over the scales that we are interested in.
The conductivity in inhomogeneous media can be rarely
simplified to a constant, as it can depend both from the direction
of the fields and currents, and vary from point to point. 
Then the above expression is just shorthand notation for the
explicit  \emph{tensorial} relationship:
\begin{equation}
  \begin{pmatrix}
    J_x \\
    J_y \\
    J_z
  \end{pmatrix}
  =
   \begin{pmatrix}
     \sigma_{xx}E_x+ \sigma_{xy}E_y+\sigma_{xz}E_z \\
     \sigma_{xy}E_x+ \sigma_{yy}E_y+\sigma_{yz}E_z \\
     \sigma_{xz}E_x+ \sigma_{yz}E_y+\sigma_{zz}E_z \\
  \end{pmatrix},
\end{equation}
where we allready used the fact that $\cond$ is simmetric when
expressed on Cartessian coordinates.
We assume that the electric field can be obtained as the
gradient of the LFP, $\phi$:
\begin{equation}
  -\grad \phi=\EF
\end{equation}
This LFP is the quantity that the microelectrode arrays (MEAs) permits us
to measure on the extracelular medium.
To obtain the sources and sink densities, we use Gauss Divergence Theorem,
stated in differencial form:
\begin{equation}
\nabla \cdot \Jd =I_m
\end{equation}
We perform the divergence operation on the current density, not
on the electric field. Only when the medium is isotropic and homogeneous
we can be sure that both operations coincide within a positive factor.

Here we shall investigate
posible influence of changing conductivity on current density
($\Jd$) and current source density ($I_m$). 


\section{Analytical  derivation of the CSD}


In experiments of this sort performed on hippocampus slices, the
MEAs permit us to measure a sample of LFP over discrete points on a
two dimensional lattice, so we do not have possibility of
calculating currents or electric fields on the third dimension of space.
The slices of the hippocampus are very thin, and we do not consider
the posibility of current going outside the neural tissue sample (in one
direction there is air and on the other the MEA), so we
can reduce our formalism to two dimensional space. We then take
$J_z=0$ and use $(x,y)$ as a cartesian coordinate system aligned
with the MEA. Then the \emph{flat} current density  is:
\begin{equation}
  \begin{pmatrix}
    J_x\\
    J_y
  \end{pmatrix}  =
  \begin{pmatrix}
    \sigma_{xx}E_x+\sigma_{xy}E_y \\
    \sigma_{xy}E_x+\sigma_{yy}E_y 
  \end{pmatrix}=
  \begin{pmatrix}
    -\sigma_{xx}\partial_x \phi -\sigma_{xy}\partial_y \phi \\
    -\sigma_{xy}\partial_x \phi -\sigma_{yy}\partial_y \phi 
  \end{pmatrix}.
\end{equation}
The divergence, in cartesian coordinates, yields:
\begin{equation}\label{csd2dexp}
  I_m=\partial_x J_x+\partial_y J_y=
  -\partial_x( \sigma_{xx}\partial_x \phi +\sigma_{xy}\partial_y \phi)
  -\partial_y( \sigma_{xy}\partial_x \phi +\sigma_{yy}\partial_y \phi). 
\end{equation}
Each one of the terms of the right hand side of eq. \ref{csd2dexp}
has the form
\begin{multline}
  \partial_j (\sigma_{jj}\partial_j\phi) +\partial_j (\sigma_{jk}\partial_k \phi)
  =  \\
  (\partial_j \sigma_{jj})(\partial_j\phi) +\sigma_{jj}\partial_j^2\phi +
  (\partial_j \sigma_{jk})(\partial_k \phi)+\sigma _{jk}\partial_j\partial_k \phi.
\end{multline}
The complete expression is cumbersome to write, but it is important to
have \emph{full} expansion in order to make the numerical considerations
in what follows.
\begin{equation}\label{Imcart}
  \begin{split}
  -I_m= & (\partial_x \sigma_{xx}) (\partial_x \phi) +
  (\partial_y \sigma_{yy}) (\partial_y \phi)\\
& +(\partial_x \sigma_{xy}) (\partial_y \phi) +(\partial_y \sigma_{xy}) (\partial_x \phi)\\
& + \sigma_{xx} \partial_x^2 \phi + \sigma_{yy} \partial_y^2 \phi\\
& + 2\sigma_{xy} \partial_x\partial_y \phi. 
\end{split}
 \end{equation}
  

\begin{figure}[h]
\centering
\begin{subfigure}[t]{0.20\textwidth}
\includegraphics[width=\textwidth]{DiagramaCAetDG02.pdf}
\caption{}
\label{diagCA}
\end{subfigure}
\quad
\begin{subfigure}[t]{0.33\textwidth}
\includegraphics[width=0.67\textwidth]{DiferentialCoordinates01.pdf}
\caption{}
\label{loceig}
\end{subfigure}
\begin{subfigure}[t]{0.20\textwidth}
\includegraphics[width=\textwidth]{PseudoCoordinates01.pdf}
\caption{}
\label{pseudocor}
\end{subfigure}
\caption{\ref{diagCA}) A diagram showing the position of
the hippocampus slice over the MEA. \ref{loceig}) 
A tangent space basis along the
eigenvectors of $\cond$. The red vectors indicate the
Cartesian Coordinates aligned with the MEA latice. 
\ref{pseudocor}) A putative local coordinate system spawned by the 
eigenvectors $(t,a)$ at every point of intereset. 
 }\label{diagdif}
\label{esquemas01}
\end{figure}


This expressions may be regardes as inescesarily long, as the 
$\cond$ tensor, being a real symmetric tensor, can be represented
in local coordinates which diagonalize it. Those coordinates have
physical meaning, as they are the resolving directions in which
conductivity behaves differently and denote some underlying structure
in the studied material. The usual assumption is that 
the conductivity tensor is isotropic but not homogeneous, and it 
changes along the apical direction, and not along the 
differentiated strata (e.g. stratum pyramidale, stratum granolosum).
Let us suppose that this is approximately correct, and that very close
to the stratum pyramidale we represent $\cond$ into the coordinates
defined by its eigenvectors (see fig. \ref{esquemas01}).
 We shall call them $(t,a)$ coordinates
and asume that the first is approximately tangent to the
strata and the other to the apical direction. The $\cond$ tensor would 
have the expression in such coordinates
\begin{equation}
\tilde{\cond}=
\begin{pmatrix}
\sigma_t & 0 \\
0 & \sigma_a
\end{pmatrix}
\end{equation} 
Then, the equation \ref{Imcart} becomes
\begin{equation}\label{Imapic}
\begin{split}
-I_m= & (\partial_t \sigma_{t}) (\partial_t \phi) +
(\partial_a \sigma_{a}) (\partial_a \phi)  \\
& + \sigma_{a} \partial_a^2 \phi + \sigma_{t} \partial_t^2 \phi, 
\end{split}
\end{equation}
as $(t,a)$ are also local ortogonal coordinates. 
One may wonder why we are explicitly stating the non-diagonal 
version of the equations. The data as it is delivered by
the measuring apparatus is on a strictly cartessian grid, and
we have to perform the numerical derivatives over this data, so
we have to know how these assumptions appear in both sets
of coordinates. Besides, the \emph{tangent-apical} coordinates
are just a teoretical tool, a numerical implementation of such a coordinate
set would be  complicated and prone to numerical errors. 


If the medium where the ions diffuse is liquid, and the structures
therein are randomly distributed, the assumption of isotropy may hold 
on average in sufficiently big scales \cite{Bedard11}. 
In order to not make the notation more cumbersome, let us aggree
that we are dealing with averages over volumes of the aproximate size
of the electrodes. Then, 
$\sigma_t=\sigma_a=\sigma$. As we stated, there is no
change along the strata, but there is change along the appical
direction:
\begin{align}
\partial_t \sigma & =0 \\
\partial_a \sigma & \neq 0. 
\end{align}
We are then left with
\begin{equation}
I_m=-\partial_t \sigma \partial_t \phi - \sigma \nabla^2 \phi.
\end{equation}
Here we must point out that the Laplacian is an invariant operator,
so the last summand of the equation could be in principle be implemented
in \emph{any} convenient set of coordinates, but in numerical aplications,
the \emph{Discrete Laplacian Operator} is \emph{not} invariant (further
discution in the following section). 
The term $\partial_t \sigma \partial_t \phi$ is the hindrance to 
ionic displacement due to dense layers of soma and dendrites near the
CA structure. Most autors have neglected its efects, and in one
dimensional analysis the structure of sources and sinks is not affected
by this. Now, as the aquisition of data in two dimensions becomes
finer, this term could lead to inexactitude in the location of the
sinks and sources, depending on the magnitude of the
derivative of the conductance. Locally both $(x,y)$ and
$(t,a)$ are orthonormal systems, so we can change from one to another 
with a rotation. If $theta$ is the angle between the $x$ vector
and $t$, the partial derivatives follow the next set of relations:
\begin{align}
\partial_t \sigma &= \cos\theta \partial_x \sigma +\sin\theta \partial_y \sigma=0\\
\partial_a \sigma &= -\sin\theta \partial_x \sigma +\cos\theta \partial_y \sigma
\end{align}

After a bit of algebraic juggling we end up with
\begin{equation}
-I_m=\sigma\nabla^2 \phi + 
\partial_x\sigma (\partial_x \phi - \tan \theta \partial_y \phi)
\end{equation} 


In order to estimate the influence
of this summand on the result, we shall make some toy models which
represent the variation of $\sigma$ along the apical direction.






For a carefull and complete derivation and discution of the CSD, 
see \cite{Bedard11}. We start with the charge conservation equation,
\begin{equation}
\nabla \cdot \Jd + \frac{d \rho}{d t} =0.
\end{equation}



\bibliographystyle{plain}
\bibliography{BiblioReportes01}



\end{document}
