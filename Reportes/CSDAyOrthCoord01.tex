\documentclass{article}

%usepackage[utf8]{inputenc}
 \usepackage{fontspec}
\usepackage{amsmath}
\usepackage[spanish]{babel}
\usepackage{graphicx}

\usepackage{caption}
\usepackage{subcaption}



 \setmainfont[Ligatures=TeX]{Oldstyle HPLHS}

\newcommand{\Jd}{\mathbf{J}}
\newcommand{\EF}{\mathbf{E}}
\newcommand{\cond}{\boldsymbol{\sigma}}
\DeclareMathOperator{\diver}{div}

\title{Densidad de Fuentes de Corriente y Coordenadas generalizadas}
\author{W. P. Karel Zapfe}

\begin{document}

\maketitle

\section{Cálculo Diferencial y Electrodinámica sobre el Hipocampo}

En principio, la densidad de fuentes de corriente se puede obtener
de la densidad de corriente por continuidad:
\begin{equation}
\nabla \cdot \Jd +\frac{\partial \rho}{\partial t}=0.
\end{equation}
Es usual llamarle a $\partial \rho /\partial t =-I_m$, la densidad de 
fuentes de corriente.
Ahora es momento para recordar que la notación ``nabla'' resulta especialmente
inconveniente para cambios de coordenadas, pues nos hace creer (y meter la pata)
que sus componentes cambian siempre igual, y eso no es el caso. Por
ello voy a reescribir de una forma más algorítmica la ecuación de arriba
de la siguiente forma:
\begin{equation}
\diver \Jd =I_m.
\end{equation}
Obsérvese que la cantidad $\Jd$ es una cantidad \emph{vectorial}, cuyos componentes
cambian de acuerdo a las coordenadas usadas localmente, sin embargo $I_m$ es un
escalar, es decir, es invariante frente a cambios de coordenadas. Por
ende, la operación $\diver ()$ tiene que contrarrestar el cambio de coordenadas,
es decir, es un operador \emph{contravariante}, a diferencia del operador
gradiente, que denotamos como $\nabla$ sin el puntido de producto interno\ldots

Si el material es ohmico y descente esperamos esta relación:
\begin{equation}
\Jd=\cond \EF,
\end{equation}
despreciando la contribución del campo de desplazamiento eléctrico (ver
las eqs. 18 y 28 los argumentos entre ellas en \cite{Bedard11}).
Por el momento supongamos que estamos en un régimen linear, donde
$\cond$ no depende de $\EF$. Sin embargo no hay ninguna razón
de entrada para asumir $\cond$ como isotrópico o homogéneo. Así que en
general consideremos a $\cond$ como un tensor, cuyas componentes generales
son:
\begin{equation}
\cond=\begin{pmatrix}
\sigma_{xx} & \sigma_{xy} \\
\sigma_{yx} & \sigma_{yy}
\end{pmatrix}
\end{equation}
Nótese que ya estoy restringiendome a un universo bidimensional.
Este tensor es real y simétrico, así que básicamente esto quiere decir
que podemos encontrar un sistema de coordendadas \emph{ortonormales}
donde el tensor sólo tenga componentes diagonales (el sistema estaría
alineado con los ``componentes principales'', véase por ejemplo la
ecuación 6 en Mitzdorf,
\cite{Mitzdorf85}). El cambio de coordenadas del sistema $(x,y)$ al
sistema de componentes principales, que denotaré por $(t',a')$ es localmente
\emph{ortogonal}, pues preserva norma y producto interno de nuestra
base cartesiana usual. Este cambio de coordenadas es local, ya que como
no hemos supuesto que $\cond$ sea constante, los ejes principales
pueden variar de punto a punto. Vamos a considerar que son 
bastante suaves y que por lo tanto dicha transformación local es la
diferencial de un cambio de cordenadas ortogonal adecuado.
Dado que la conductividad, a pesar de estar representada por este
objeto matemático abstracto (un tensor de rango dos), es algo muy concreto,
la sabiduría convencional indica que los ejes principales en cada punto
deben de estar alineados con las estructuras notorias de nuestro material.

En nuestro caso particular, las estructuras de interés son los cuerpos laminados
donde se encuentran los somas de las neuronas, el ``estrato piramidal'',
y las capas dendríticas cerca de éste. Podríamos imaginar entonces
que los ejes principales siguen, punto por punto, a lo largo del
CA, un vector tangente al centro de este estrato piramidal y otro vector
normal a éste, que llamare ``apical'' por analogía, véase la fig. 
\cite{esquemas01}.

\begin{figure}[h]
\centering
\begin{subfigure}[b]{0.45\textwidth}
\includegraphics[width=\textwidth]{DiagramaCAetDG02.pdf}
\caption{Esquema}
\label{diagCA}
\end{subfigure}
\quad
\begin{subfigure}[b]{0.45\textwidth}
\includegraphics[width=\textwidth]{PseudoCoordinates01.pdf}
\caption{Coordenadas}
\label{pseudocor}
\end{subfigure}
\caption{ (\ref{diagCA}) Esquema del hipocámpo en nuestras rebanadas y su
colocación aproximada en el BioCAM. Se muestran los estratos somáticos.
(\ref{pseudocor}) Coordenadas ortogonales putativas, a lo largo del eje
``tangente al estrato piramidal'', $t$ y ``apical'', $a$. } 
\label{esquemas01}
\end{figure}

Obsérvese que para el argumento no es central el hecho de que estas coordenadas
estén realmente alineadas con la dirección apical y el estrato piramidal, 
lo único importante es que existen ejes
principales para la conductividad en cada punto, y que la transformación
localmente puede ser vista como una rotación (ya que los eigenvectores pueden
escogerse normales). Voy  a denotar por un tilde a los objetos representados
en las coordenadas de los ejes principales, y por $T$ 
a la transformación diferencial:
\begin{equation}
\tilde{\cond}=
\begin{pmatrix}
\sigma_{t} & 0 \\
0 & \sigma_{a}
\end{pmatrix} = T \cond T^{-1}
\end{equation}

\begin{equation}
  \tilde{\Jd} = \tilde{\cond}\tilde{\EF}= T\Jd= (T \cond T^{-1}) T \EF = T\cond \EF
\end{equation}


 

\bibliographystyle{plain}
\bibliography{BiblioReportes01}



\end{document}