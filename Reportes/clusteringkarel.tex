As an extension to the trajectory concept in the CSD-CM we
shall propose a more general concept: spatio-temporal clustering of the CM.
This extends the notion of trajectory to more general grouping of the
CM data.

In view ot the necesity of clasifing the CM by anatomical location,
the need of an automatic, unsupervised clasification of the various CM
over the underlying physiological structure was possed.  The CM data
consists of 4 continous numerical dimenisons and a binary categorical
one. These are the 2 dimensions of of spatial location, which are constrained
to the dimensions of the MEA, the temporal variable (which can be
considered discrete if we use sampling frames as time unit), and the intensity of the CM.
The sign of the last one can be considered a separate categorical variable which
has the values ``Sink'' or negative, and ``Source'' or positive. 
Our problem was then to use the numerical variables to group the CM data in
a way that could sheed some light over the anatomical structure that
generates it. In the figure \ref{puntostodos} we present the CM data
in the MEA coordinates obtained from an experiment with facilitated activity
during a burst of spikes ( around 285 ms of recording ). A visual inspection
seems to suggest a clustering of the points in layers and segments over and near the
Cornu Ammonis Structure, with fewer points over other structures. Most interesting
is the three layered clustering that seems to be around CA3 and the two layeered
structure in CA1. 

In the figure \ref{puntosconcolor} we also colorize the dost by temporal emergence.
This also reveals that the layering has also a reccurent order of appearance. Intensity
layering was also explored, but heuristic reasons based on previous trajectory
analysis put doubts on the validity of this clasification. Namely, we expect
that the genaralization of trajectories to also have a rising and fading behaviour,
and thus the intensity should not be used to sepparate different cluster, but to
find regular activity inside each cluster. It could be argued that intensity also
classifies multiunitary activity by type and number of putative units, but then,
this classification would require using intervals of intensity instead of mere
points and would complicate the analysis considerably.

We decided to  use only the spatiotemporal variables for classification. 
The temporal variable has different units, so a prudent normalization is due
in order to not to give to it a preponderant or negligible weight in relation to the
spatial ones. In absence of any other information, the sensible thing to do is to
scale it so that its maximum span is of the same order of magnitude to the
spatial length of the data, namely 64 arbitrary units, which is the convenient unit
of measurement of the MEA ( 64 electrodes correspond to 2.27 mm).
With this normalizazion, the data of the CM exist in a 3 dimensioanl cube of 64 a.u.
of length on each dise. We can then procede to cluster the data acording to
certain criteria.

Clustering algorhytms vary not only in implementation, but in design criteria.
Thus they produce different results. We had to test a few of them. Our choice at the
end was the so called DBSCAN (density based spatial clustering of applications with noice).
This algorhytm is conceptually clean and can produce heteregenous clusters, in contrast
to others such as HDC ( Hierarquical Data clustering). The DBSCAN clustering depends
on two parameters: a radio of neighboorhood and a minimal number of neighbours. Each
point that is demed to be part of a certain cluster has to have that number of neighbours
that are also part of the same set inside an sphere of the choosen radius. These
parameters can be choosen on physioligical grounds altough we can device a more
abstract way of optimizing them. 

We decided to use a certain ``esthetic'' criterium for choosing the optimal parameters
for the DBSCAN algorythm.  It is its accordance with Zipf's Law.  This could post out
the accordance with the allometric inverse algebraic laws that are observed in
anatomical studies, but it is, in the end, a somewhat arbitrary criterium. Even thou, a good
fit with Zipfs Law could be an indictor of a natural clustering of data points.
A problem that appeard using this criterium is that it is such a pervarsive
behaviour, that it gives a good fit to allmost any set of reazonable parameters
that one encounters. So we had to come up with criteria for what these ``reasonable
parameters'' may be.  We settled that we shouldn't have to many unasigned points, and
that no cluster should be too big, or that there were to many clusters. 
