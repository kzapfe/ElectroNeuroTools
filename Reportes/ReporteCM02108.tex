karel@mezcalina Reportes $more Reporte01.tex 
\documentclass{article}

\usepackage[utf8]{inputenc}
\usepackage[spanish]{babel}
\usepackage{graphicx}

\title{Reporte: Centro de Masa para potenciales evocados}


\author{W. P. Karel Zapfe}

\begin{document}

\maketitle

\section{Los Datos}

Para llevar a cabo está parte del trabajo, se me proporcionaron
los siguientes datos experimentales.
Dos experimentos largos de estimulación ortodrómica y uno
antidrómica. Cada experimento largo fue llevado a cabo
en dos etapas, una etapa control y una etapa con biliculina.
Cada etapa tiene estimulos de varios voltajes, en orden creciente,
de 1V a 9V.
El experimento antidrómico tiene una etapa control, una etapa con
mefluoquina y una etapa con nbqx-apv. La sucesión de estímulos es la misma.

El segundo experimento ortodrómico resultó inútil para los propósitos
de este análisis, dado que en todas las estimulaciones el número de
canales saturados sobre la región de interés (de ahora en adelante ROI)
era excesivo, y obfuscaban el cálculo densidad de fuentes de corriente
(de ahora en adelante CSD), así que nos concentramos en el primero experimento.
De ahí las estimulaciónes con 5V y 6V resultarón tener el registro más
limpio en condiciones de control, y la de 5V y 9V para la etapa
con biliculina.

El experimento antidrómico tuvo los mejores registros en 5V para control,
y en 7V para ambas drogas. 

\section{Análisis exploratorio}

El primer paso para llevar a cabo el análisis de CSD-CM (Centro de Masa
para Densidades de Fuentes de Corriente) es conocer la posición de
los cuerpos anatómicos de interés en el registro, y simultaneamente,
conocer cuantos y cuales electrodos muestran respuestas fisiológicas y
cuales están saturados y resultan inútiles para nuestro propósito.
Una umbralización de Otsu sobre la variancia de cada canal 1 ms después del
estímulo nos da una indicación de posibles sedes para los somas de las neuronas.
complementado con la lista de los electrodos que registraron actividad fisiológica, podemos establecer la posición de CA y GD sobre el MEA. Una foto
podría ayudar de estar disponible, pero no cuento con ella, así que
tendremos que descanzar sobre los trazos fisiológicos y nuestro
conocimiento de la forma de la rebanada a priori. Muestro
los resultados así obtenidos para los casos analisados.


\section{ḱCSD y Regularización de Tikhonov}

Este tipo de experimento suele tener varios electrodos saturados, y pueden
estar sobre o muy cerca de la ROI. Un cálculo de diferencia finita para
obtener la CSD como hemos llevado a cabo anteriormente resulta muy inconveniente. Un solo electrodo ruidoso arruina a sus 8 vecinos en general, haciendo
que perdamos una parte sustancial del registro. En cambio, un análisis
de tipo inverso, tal como el kCSD \cite{Potworowski2011} nos da un resultado
mucho más suave y claro, con el costo de mayor tiempo computacional y unas
suposiciones extra. He decido implementer el proceso de kCSD para
registros masivos tales como los que proporciona el BIOCAM, paralelizando
sobre CPU el algorítmo, dado que es un problema trivialmente paralelizable.
Contamos asimismo con una versión parelela en GPU mucho más veloz
(desarrollada por Néstor Castillo para su tésis de licenciatura), pero
que requiere tarjetas gráficas nVidia de buena calidad para ser usada.

El kCSD requiere que hagamos una suposición razonable sobre
la forma de la función fuente-pozo cerca de un electrodo. La forma especifica
de la función no es importante mientras cumpla ciertas características
de regularidad y vaya acorde con nuestro conocimiento de como se puede
generar un espacio de funciones. He tomado como función modelo
una función característica circular con un radio de
media distancia inter electrodo (21 \mu m).















\section{}


\end{document}
