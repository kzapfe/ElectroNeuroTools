\documentclass{article}

\usepackage[utf8]{inputenc} 
\usepackage[spanish]{babel}
\usepackage{graphicx}
\usepackage{subcaption}
\usepackage{textcomp}


\title{Reporte: Centro de Masa para potenciales evocados}


\author{W. P. Karel Zapfe}

\begin{document}

\maketitle

\section{Los Datos}

Para llevar a cabo está parte del trabajo, se me proporcionaron
los siguientes datos experimentales.
Dos experimentos largos de estimulación ortodrómica y uno
antidrómica. Cada experimento largo fue llevado a cabo
en dos etapas, una etapa control y una etapa con biliculina.
Cada etapa tiene estímulos de varios voltajes, en orden creciente,
de 1V a 9V.
El experimento antidrómico tiene una etapa control, una etapa con
mefluoquina y una etapa con nbqx-apv. La sucesión de estímulos es la misma.
Cada estímulo fue repetido 3 veces. He promediado los datos
centrados en el pico del estímulo. 

El segundo experimento ortodrómica resultó inútil para los propósitos
de este análisis, dado que en todas las estimulaciones el número de
canales saturados sobre la región de interés (de ahora en adelante ROI)
era excesivo, y ofuscaban el cálculo densidad de fuentes de corriente
(de ahora en adelante CSD), así que nos concentramos en el primer experimento.
De ahí las estimulaciones con 5V y 6V resultaron tener el registro más
limpio en condiciones de control, y la de 5V y 9V para la etapa
con biliculina.

El experimento antidrómico tuvo los mejores registros en 5V para control,
y en 7V para ambas drogas. 

\section{Análisis exploratorio}

El primer paso para llevar a cabo el análisis de CSD-CM (Centro de Masa
para Densidades de Fuentes de Corriente) es conocer la posición de
los cuerpos anatómicos de interés en el registro, y simultáneamente,
conocer cuantos y cuales electrodos muestran respuestas fisiológicas y
cuales están saturados y resultan inútiles para nuestro propósito.
Una umbralización de Otsu sobre la varianza de cada canal 1 ms después del
estímulo nos da una indicación de posibles sedes para los somas de las neuronas.
complementado con la lista de los electrodos que registraron actividad fisiológica, podemos establecer la posición de CA y GD sobre el MEA. Una foto
podría ayudar de estar disponible, pero no cuento con ella, así que
tendremos que descansar sobre los trazos fisiológicos y nuestro
conocimiento de la forma de la rebanada a priori. Muestro
los resultados así obtenidos para los casos analizados en la figura \ref{exploratorio},
y una muestra de los trazos en canales activos en la figura \ref{trazosejemplo1}.
Los datos encontrados son poco esperanzadores. La estimulación ortodrómica parece
no haber causado mucho registro en GD, a diferencia de la antidrómica, en donde
podemos apreciar actividad en ambas estructuras.


\begin{figure}[h]
    \centering
    \begin{subfigure}[b]{0.45\textwidth}
        \includegraphics[width=\textwidth]{LFP_Promedio_est_con5-mapa.png}
        \caption{Orto, Control, 5V}
        \label{ocon5}
    \end{subfigure}
    ~ %add desired spacing between images, e. g. ~, \quad, \qquad, \hfill etc. 
      %(or a blank line to force the subfigure onto a new line)
    \begin{subfigure}[b]{0.45\textwidth}
        \includegraphics[width=\textwidth]{LFP_Promedio_est_con6-mapa.png}
        \caption{Orto, Control, 6V}
        \label{ocon6}
    \end{subfigure} \\
 \begin{subfigure}[b]{0.45\textwidth}
        \includegraphics[width=\textwidth]{LFP_Promedio_est_bic5-mapa.png}
        \caption{Orto, Bic, 5V}
        \label{ocon5}
    \end{subfigure}
    ~ %add desired spacing between images, e. g. ~, \quad, \qquad, \hfill etc. 
      %(or a blank line to force the subfigure onto a new line)
    \begin{subfigure}[b]{0.45\textwidth}
        \includegraphics[width=\textwidth]{LFP_Promedio_est_bic9-mapa.png}
        \caption{Orto, Bic,  9V}
        \label{ocon6}
    \end{subfigure} \\
 \begin{subfigure}[b]{0.45\textwidth}
        \includegraphics[width=\textwidth]{LFP_Promedio_310816_stim_antidromic5-mapa.png}
        \caption{Anti, Control, 5V}
        \label{ocon5}
    \end{subfigure}
    ~ %add desired spacing between images, e. g. ~, \quad, \qquad, \hfill etc. 
      %(or a blank line to force the subfigure onto a new line)
    \begin{subfigure}[b]{0.45\textwidth}
        \includegraphics[width=\textwidth]{LFP_Promedio_310816_stim_antidromic_nbqx_apv7-mapa.png}
        \caption{Anti, nbqxapv, 7V}
        \label{ocon6}
    \end{subfigure} \\
    \caption{Análisis Exploratorio para los datos obtenidos,
      se muestra un cuadro correspondiente a 2.58 ms después del estimulo,
      en varios casos. Podemos ver que la orientación del experimento Antidrómico
      es diferente (una reflexión horizontal, prácticamente) respecto al Ortodrómica.
    }\label{exploratorio}
\end{figure}


\begin{figure}[h]
  \centering
  \includegraphics[width=0.95\textwidth]{TrazosEjemplares_bic5.png}
  \caption{Trazos ejemplares del conjunto de canales detectados como
    activos automáticamente, usando el experimento de Biliculina a 5V, ortodrómica.}
  \label{trazosejemplo1}
\end{figure}



\section{kCSD y Regularización de Tikhonov}

Este tipo de experimento suele tener varios electrodos saturados, y pueden
estar sobre o muy cerca de la ROI. Un cálculo de diferencia finita para
obtener la CSD como hemos llevado a cabo anteriormente resulta muy inconveniente. Un solo electrodo ruidoso arruina a sus 8 vecinos en general, haciendo
que perdamos una parte sustancial del registro. En cambio, un análisis
de tipo inverso, tal como el kCSD \cite{Potworowski2011} nos da un resultado
mucho más suave y claro, con el costo de mayor tiempo computacional y unas
suposiciones extra. He decido implementar el proceso de kCSD para
registros masivos tales como los que proporciona el BIOCAM, paralelizando
sobre CPU el algoritmo, dado que es un problema trivialmente paralelizable.
Contamos asimismo con una versión paralela en GPU mucho más veloz
(desarrollada por Néstor Castillo para su tésis de licenciatura), pero
que requiere tarjetas gráficas nVidia de buena calidad para ser usada.

El kCSD requiere que hagamos una suposición razonable sobre
la forma de la función fuente-pozo cerca de un electrodo. La forma especifica
de la función no es importante mientras cumpla ciertas características
de regularidad y vaya acorde con nuestro conocimiento de como se puede
generar un espacio de funciones. He tomado como función modelo
una función característica circular con un radio de
media distancia inter electrodo (21 \textmu m). En los casos que he tratado,
hasta ahora no parece haber diferencia considerable si cambiamos la
función característica por una gaussiana o disminuimos o aumentamos el radio
por un factor de dos, lo cual es una buena noticia, dado que quiere decir
que nuestros resultados son estables frente a los supuestos que estamos haciendo.

Los operadores del kCSD son dos: un operador normalmente designado por $K$ que representa
la transformación del espacio de Fuentes al espacio de Voltajes, y un operador
$\tilde{K}$ que representa la proyección de las fuentes sobre un espacio de funciones ideales del mismo espacio de fuentes (la base del espacio). El cálculo de estos operadores
es la parte computacionalmente intensiva de la rutina, pero sólo tiene que hacer una vez
por caso (i.e. por estímulo), dado que el cambio depende de cuales sean los electrodos
inservibles en cada conjunto de datos.

La concatenación directa de estos operadores presupone una señal ideal, sin ruido.
En el mundo real la obtención del LFP es un proceso ruidos, en nuestro caso
tenemos una razón de ruido a señal de aproximadamente 1/10.  Para disminuir el efecto
del ruido en la aplicación del kCSDA se sobreestima la contribución de los canales
vecinos ligeramente, de forma que el ruido del canal en cuestión quede
aplacado por la suma de varias señales ligeramente diferentes. Encontrar
la conveniente de aplacamiento requiere probar un parámetro sobre valores
razonables y llevar a cabo validación cruzada.
Valores muy bajos dan señales muy ruidosas, mientras que señales demasiado grandes
suavizan en exceso la señal.
Es bueno hacer la comparación contra un ejemplo donde podamos obtener la CSD
por el método de diferencia finita, así que tengamos un ejemplo contra
el cual comparar primero. He encontrado que un valor adecuado para el
parámetro de regularización es 0.05. Los resultados para un ejemplo
tomado de un pequeño recuadro de 24x24 electrodos se
muestran el la figura \ref{regultik}.

La regularización de Tikhinov basta hacerla una vez por cada nivel de
ruido, es decir, si confiamos en que todos los experimentos evocados
mantienen aproximadamente la misma razón ruido/señal, una vez
bastará. Mas adelante podemos hacer una prueba más limpia.

Una vez obtenido el parámetro adecuado, podemos obtener la CSD
de cada conjunto de datos si ya contamos con los operadores
$K$ y $\tilde{K}$.

\begin{figure}[h]
  \includegraphics[width=0.95\textwidth]{VarioskCSD03.png}
  \caption{Prueba de regularización de Tikhonov}\label{regultik}
\end{figure}


\section{Componentes Disjuntos.}

Obtenida la CSD del kCSDA, podemos empezar a separar los componentes disjuntos
de pozos y fuentes. La rutina que se había utilizado anteriormente requirió
un poco de ajuste, dado que las unidades arbitrarias del kCSDA reultan diferentes
después de la regularización. He creado una función que toma como argumento
un ``cero gordo'' alrededor del valor neutro de la CSD, y este debe
ser ajustado respecto a la escala que tengamos. He encontrado que un valor
de 0.01 veces el valor máximo de una fuente o un pozo es un buen criterio
para separar los componentes. En las figuras \ref{cmcontrol} y \ref{cmbic}
muestro
ejemplos de componentes disjuntos para varios tiempos después del estimulo,
con y sin biculina. Existe una regularidad en los componentes que permite
discernir la punta de CA3. 

\begin{figure}
  \centering
    \begin{subfigure}[b]{0.95\textwidth}
        \includegraphics[width=\textwidth]{DisjointComponents_Evento_control-5-120.png}
    \end{subfigure}
    \\
     \begin{subfigure}[b]{0.95\textwidth}
        \includegraphics[width=\textwidth]{DisjointComponents_Evento_control-5-127.png}
    \end{subfigure}
     \\
      \begin{subfigure}[b]{0.95\textwidth}
        \includegraphics[width=\textwidth]{DisjointComponents_Evento_control-5-134.png}
    \end{subfigure}
      \\
       \begin{subfigure}[b]{0.95\textwidth}
        \includegraphics[width=\textwidth]{DisjointComponents_Evento_control-5-141.png}
    \end{subfigure}
       \caption{Componentes Disjuntos para varios momentos después del estímulo, condiciones
         de control, ortodrómica, 5V}\label{cmcontrol}
\end{figure}

\begin{figure}
  \centering
    \begin{subfigure}[b]{0.95\textwidth}
        \includegraphics[width=\textwidth]{DisjointComponents_Evento_bic-5-120.png}
    \end{subfigure}
    \\
     \begin{subfigure}[b]{0.95\textwidth}
        \includegraphics[width=\textwidth]{DisjointComponents_Evento_bic-5-127.png}
    \end{subfigure}
     \\
      \begin{subfigure}[b]{0.95\textwidth}
        \includegraphics[width=\textwidth]{DisjointComponents_Evento_bic-5-134.png}
    \end{subfigure}
      \\
       \begin{subfigure}[b]{0.95\textwidth}
        \includegraphics[width=\textwidth]{DisjointComponents_Evento_bic-5-141.png}
    \end{subfigure}
       \caption{Componentes Disjuntos para varios momentos después del estímulo, condiciones
         con biculina, ortodrómica, 5V}\label{cmbic}
\end{figure}

\section{Trayectorias}

Finalmente comparemos las trayectorias de Centros de Masa para el experimento ortodrómico
a 5V en condiciones control y biculina. Esta representación puede develar mejor los cambios
de acción grupales en las estructuras fisiológicas \ref{traymedreqs}.


\begin{figure}
  \centering
    \begin{subfigure}[b]{0.45\textwidth}
      \includegraphics[width=\textwidth]{LFP_Promedio_est_con5-Tray-t-134-medreqs.png}
        \caption{control}
    \end{subfigure}
    ~
    \begin{subfigure}[b]{0.45\textwidth}
       \includegraphics[width=\textwidth]{LFP_Promedio_est_bic5-Tray-t-134-medreqs.png}
       \caption{biculina}
     \end{subfigure}
     \caption{Trayectorias con requisitos de aceptación intermedios, esto es,
       cuya longitud sea mayor a 8 cuadros (correspondiente a poco mas de 1ms) y
       8 u.a. de densidad.}
\end{figure}


\end{document}
