\documentclass{beamer}


\usepackage{amsmath}
\usepackage[utf8]{inputenc}
\usepackage[spanish]{babel}
\usepackage{color}
\usepackage{ragged2e}

\justifying

\decimalpoint

\usefonttheme{serif}

\newcommand{\dd}{\, \mathrm{d}}
\newcommand{\xq}{\, \mathbf{x}}

\begin{document}


\begin{frame}
  \frametitle{De las medidas eléctricas a lo subyacente:} 
  \begin{equation}
    \varphi(\xq)
    =\int \frac{1}{4\pi\epsilon}
    \frac{\rho(\xq')}{|\xq-\xq'|} \dd^3 \xq'  
  \end{equation}
    \begin{equation}
      \frac{\partial \rho}{\partial{t}}(\xq)=
      -\frac{\sigma}{\epsilon}\rho(\xq)
    \end{equation}
  \begin{center}
  \begin{tabular}{ccc}
   \textcolor{red}{LFP} & 
   \includegraphics[width=0.3\textwidth]{Decaimiento01.pdf} &
   \textcolor{blue}{CSD}
  \end{tabular}
  \end{center}
  {\tiny Nicholson \emph{et} Freeman , J.   Neurophys. 38, 356 (1975)\\
         Bédard \emph{et} Destexhe, PRE 84, 041909 (2011) }
\end{frame}





\begin{frame}
\frametitle{Un cero bien localizado.} 
Tanto en la densidad de cargas, como en la densidad de fuentes de corriente
hay una distinción clara entre \emph{pozos} y \emph{fuentes}.
No es el caso para el LFP, que mide las cosas a partir de una ``tierra''
arbitraria.

\begin{tabular}{cc}
      \includegraphics[width=0.4\textwidth]{VectoresRojos_732.png} &
    \includegraphics[width=0.55\textwidth]{CSD_SeminarioJunio-cb-732.png} 
\end{tabular}
\end{frame}


\begin{frame}
  \frametitle{Promedios Vectoriales o \emph{Centros de Masa}}
  \begin{equation}
    \langle \xq(t) \rangle=\sum_k m(\xq_k)\xq/\sum_k m(\xq_k)
  \end{equation}
  
  \begin{center}
    \includegraphics[width=0.75\textwidth]{ImagenManjarrez01.jpg}
  \end{center} 
  
 \begin{flushright}
  {\tiny  Manjarrez \emph{ et all}, Brain Research, 1145, 239 (2007) \\
      Chao \emph{ et all}, Neuroinformatics, 3, 263 (2005) }
  \end{flushright}

\end{frame}


\begin{frame}
\frametitle{Centros de Masa en Geometrias Cóncavas} 
\begin{center}
  \begin{tabular}{cc}
    \includegraphics[width=0.45\textwidth]{ImagenGizMultiColorChica01.png} &  
    \includegraphics[width=0.45\textwidth]{DiagramaCAetDGCSDEspanol01.png}   
   % \includegraphics[width=0.55\textwidth]{CSD_CM_SeminarioJunio-732.png}
  \end{tabular}
\end{center}
\end{frame}


\begin{frame}
  \frametitle{Centros de Masa en Densidades de Fuentes}
  Podemos dividir cada conjunto (pozos o fuentes) en sus componentes
  disjuntas.
\begin{center}
  \begin{tabular}{cc}
    \includegraphics[width=0.55\textwidth]{CSD_SeminarioJunio-732.png} &  
    \includegraphics[width=0.35\textwidth]{ComponentesDisjuntosCSD01.png}   
   % \includegraphics[width=0.55\textwidth]{CSD_CM_SeminarioJunio-732.png}
  \end{tabular}
\end{center}
\end{frame}



\begin{frame}
  \frametitle{Un conjunto instantaneo de CM}
  El CM de cada \emph{Componente Disjunto} indica un putativo ``centro de actividad'' rastreable,
  localizado mucho más cerca de donde
  la señal es más fuerte. Esto es interpretable como una serie de ``focos''
  activos localizados de mayor actividad.
  \begin{center}
   \includegraphics[width=0.65\textwidth]{CSD_SeminarioJunio-900.png}   
   \end{center}
\end{frame}

\begin{frame}
  \frametitle{Trayectorias de los CM}
  Usando un algorítmo que conecte los CM cuadro por cuadro comienzan a
  formarse \emph{``trayectorias''} 
  con precisión de 1/7022 de segundo.
\begin{center}
   \includegraphics[width=0.67\textwidth]{TrayectoriasLimpiasNumeros01.png}   
   \end{center}
\end{frame}  

\begin{frame}
  \frametitle{Posibles implicaciones para el conectoma funcional}

  De esta serie de datos de alta densidad, ordenados por aparición,
  intensidad, y criterio anatómico, debe ser posible descubrir conexiones
  funcionales dentro de CA3. El formalismo es general, es aplicable
  a otras estructuras.

  (Estos ejemplos fueron creados usando actividad epileptiforme inducida por
  4AP.)
  \end{frame}
    

\end{document}
