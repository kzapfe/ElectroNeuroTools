\documentclass{article}

%\usepackage[utf8]{inputenc}
\usepackage{fontspec}
\usepackage{amsmath}
\usepackage[spanish]{babel}

\usepackage{graphicx}

\usepackage{caption}
\usepackage{subcaption}



\setmainfont[Ligatures=TeX]{Oldstyle HPLHS}

\newcommand{\Jd}{\mathbf{J}}
\newcommand{\EF}{\mathbf{E}}
\newcommand{\cond}{\boldsymbol{\sigma}}
\DeclareMathOperator{\diver}{div}
\DeclareMathOperator{\grad}{grad}

\title{Criterios de Conexión causal entre GD y CA}
\author{F. Ortiz,  WPK.Zapfe,  MA. Aviño }

\begin{document}

\maketitle


\section{Introduction}

En este momento, contamos con datos de alta densidad temporal
y espacial de actividad eléctrica de rebanadas ventrales del 
hipocampo de las ratas. Nuestras medidas son potenciales eléctricos
medidos en el medio extracelular inmediato a las estructuras de nuestr
interés, es decir, el giro dentado (GD) y el \emph{cuerno de Amón} (CA). 
En ese medio extracelular es posible detectar, de forma indirecta,
los potenciales de acción, cambios abruptos en el potencial
eléctrico, acción del disparo de las neuronas. Un criterio \emph{ad hoc}
nos permite establecer un momento temporal asociado al pico
del potencial de acción. La pregunta fundamental es la siguiente:
¿Qué partes del GD hacen disparar a qué partes del CA? 
Pragmáticamente esperariamos que un disparo detectado en GD fuera seguido
por uno en CA después de un intervalo de tiempo, entre 2ms  a 5 ms.  
Si con nuestro criterio de detección de picos encontramos dichos eventos
consecutivos, ¿qué tan significativo son como criterio de relación causal?
Aquí mostramos tres gráficas con criterios de significancia distinta y 
discutimos brevemente sus virtudes y defectos.

\section{Los disparos \emph{eco}}

Para no empantanar ciertos términos, vamos a describir un putativo
potencial de acción en CA como \emph{eco} si ocurre entre 
2 y 5 milisegundos después que un disparo de GD. Este término
no implicará ninguna causalidad, simplemente que ocurre dentro
del intervalo de tiempo esperado para una conexión sináptica.
Para darle un significado de conectiviad plausible, necesitamos
un criterio que establezca un nivel de causalidad entre los disparos.
Los disparos eco como medida absoluta pueden ya indicar una cierta
relación de este tipo entre GD y CA, así que proponemos, antes
que nada, observar la gráfica de ecos meramente y ver si tiene
algun sentido dentro de la fisiología ya conocida. En la figura 
\ref{EcosTotales} presentamos dicha gráfica. 

\begin{figure}[h]
\centering
\includegraphics[width=0.85\textwidth]{ConexionesEcos01.png}
\caption{ Los disparos ecos representados como lineas entre
los electrodos involucrados. Los puntos violeta representan 
electrodos debajo del GD, mientras que los negros
debajo de CA. El grosor y nivel de rojo de las lineas indican
el número de ecos totales encontrados. La linea mas gruesa
representa 210 ecos, mientras que la más delgada uno solo. 
Nótese lo poco homogeneo de la distribución. Los electrodos
que no registraron ecos no aparecen. 
}\label{EcosTotales}
\end{figure}

Notemos de esta primera figura lo poco homogeneo de la distribución.
Mientras que hay 6 electródos del GD fuertemenente conectados
a diversas regiones de CA, la mayoría de los puntos registran muy pocos
ecos, o activiad de algun tipo. Particularmente hay un electrodo en
la región infrapiramidal de GD que muestra altísima conectividad con
la región media y distal de CA3c, con una buena parte de CA3b y el
centro de CA3a. De igual forma podemos apreciar un grupo de vecinos
en la frontera entre la cresta y la región infrapiramidal que
muestra también una conexión similarmente fuerte a 
las mismas regiones. El último electrodo de la región infrapiramidal
Debajo de la cresta también encontramos un gripo de tres electrdoos
fuertemente conectados con las mismas regiones. En esta gráfica
puede haber muchos ecos casuales, es decir, falsos positivos,
dada la actividad notoria de todo CA, aún cuando GD parece 
estar silente. En cierta forma, esta imagen no está adecuadamente
normalizada a la actividad ``promedio'' de una región u otra y
de ahí la gran variedad de pesos que encontramos. Aun así, esta
gráfica tiene información que puede ser analisada formalmente y
dar resultados significativos. En particular convendría 
hacer un análisis de la distribución de las conexiones así establecidas.
Es conocimiento común que las redes que ocurren en fenómenos biológicos
deben de presentar ciertas características generales, como son 
robustez (estabilidad frente a perturbaciones o fallas), redundancia
(nodos que duplican o refuerzan conexiones), y ser \emph{libres de escala}.
Esto hace que sus conexiones sigan cierta distribución estadística
expresable como una ley de potencias con exponentes entre 1.4 y 3.
Deberíamos de revisar si es el caso en esta primera aproximación. 

La siguiente figura (fig. \ref{Jacard} muestra los ecos 
después de  una normalización que propuso
la profesora Aviño, cuyo resultante es conocido como 
``coeficiente de similiaridad de Jaccard'', y escencialmente 
intenta medir que tan iguales son dos conjuntos a su intersección. 
En este caso estamos contando los ecos como una intersección de eventos
entre GD y CA. A pesar de que este coeficiente ha sido usado en la litaretura
con propósitos similares (), admitimos que encontramos su aplición un 
tanto dudosa para nuestro caso por lo siguente. Dada la gran actividad
que presenta CA en relación a GD, estamos dividiendo el número
de ecos por números muy  grandes que representan actividad que no esta
relacionada de forma alguna con la conexión que estamos buscando. Dicho de 
otra forma: interpretar los disparos ecos como una intersección de
eventos entre dos posibles conjuntos de otros eventos no parece
tener mucho sentido. De todas formas  hay que ver que posibles
detalles revela la figura y ver su posible implicación fisiológica.

\begin{figure}[h]
\centering
\includegraphics[width=0.85\textwidth]{ConexionesJacard01.png}
\caption{ Los ecos divididos entre el la suma de disparos
\emph{sin eco} entre los electrodos relacionados. De igual forma
lineas más gruesas y más rojas indican valor más alto, sin embargo,
la escala es diferente, dado que aquí el mayor valor posible sería uno.
En esta gráfica los grosores están exagerados, dado que el el 
coeficiente de Jaccard da valores muy bajos (del orden de milésimas).
}\label{Jacard}
\end{figure}

Es notorio en esta segunda figura que los mismos electrodos muestran
la misma conectividad alta que en los ecos crudos, sin embargo el 
peso relativo que tienen es mucho más homogeneo. A su vez otros pares
de electrodos cobran un peso mucho mayor, debido a la baja cantidad
de picos registrados en ellos, en particular está el par que une
a un elemento de la hoja suprapidamidal con el centro de CA3a. 
Sin embargo no debemos de concederle aún excesivo valor a esto
sin corroborar con otra serie de datos, pues puede ser un 
artefacto debido a la baja actividad detectada en ambos sitios. 
Por otro lado el tercer electrodo de derecha a izquierda de la región
infrapiramidal muestra aún una conexión fuerte con el centro de 
CA3a. Esta en particular puede ser ya una pista importante
sobre cierto grupo de conexiones operando en estas regiones. Otra
cosa que podría decirse de esta figura es que, siendo la normalización
tan baja, permite una resolución visiual más clara en la 
estructura de abaníco formada por los electrodos más conectados.

Finalmente Karel propuso normalizar el número de ecos entre el número
de \emph{ecos posibles}, esto es, el número menor entre los disparos
de un electrodo en GD y otro en CA. Típicamente este resulta ser
el correspondiente a GD, pero no siempre. Dado que no puede haber más 
``ecos'' que el número menor de disparos en un par de electrodos, 
esto nos normaliza los ecos a un valor entre 0 y 1 donde 0 es
``ningun eco'' y 1 es ``todos los disparos tienen un eco o son un eco''.
La figura \ref{MinimalCrit} muestra este criterio de peso en la
gráfica.

\begin{figure}[h]
\centering
\includegraphics[width=0.85\textwidth]{ConexionesPseudoCausales01.png}
\caption{ Los ecos divididos entre el la suma de disparos
\emph{sin eco} entre los electrodos relacionados. De igual forma
lineas más gruesas y más rojas indican valor más alto, sin embargo,
la escala es diferente, dado que aquí el mayor valor posible sería uno.
En esta gráfica los grosores están exagerados, dado que el el 
coeficiente de Jacard da valores muy bajos (del orden de milésimas).
}\label{MinimalCrit}
\end{figure}

La figura \ref{MinimalCrit}
es notoriamente diferente de las otras dos. En particular
muestra una conexión mucho más homogenea de todo el GD a la 
región CA3c, y muestra conexiones muy fuertes en la cresta del GD 
hacia varios electrodos de CA3c y la frontera con CA3b, conexiones
que aparecen de muy bajo peso en las dos figuras anteriores.
De la misma forma que la linea más pesada de la figura \ref{Jacard}
puede deberse a un número muy bajo de registros, lo mismo puede
suceder aquí, y no debemos de confiar en estos valores hasta no 
corroborar con mayores datos. Este criterio podría, tal vez, 
revelarnos conexiones discretas, sin inferencias de peso sobre 
las diferentes regiones, pero para ello necesitariamos muestras
con mayores datos del GD. Afortunadamente el problema del procesamiento
de los datos para buscar pares de ecos ya se encuentra bastante 
más avanzado, y podemos analizar juegos de datos mucho mayores en menor
tiempo, así que el problema consiste ahora en encontrar mejores
métodos para detectar actividad espontanea en el giro dentado. Por
otra parte, aquí no aparecen considerados otro tipo de respuestas
diferentes al potencial de acción como eco. 



\end{document}
