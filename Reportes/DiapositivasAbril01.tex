\documentclass[xetex,mathserif,serif]{beamer}



\begin{document}
  \begin{frame}
    \frametitle{CSDA}
    Our measurements would allow us to obtain the \emph{Current Source Density}
    of the electric field potentials  recorded in the experiments.
    The fundamental equation which relates this two quantities is
    Gauss Law for an ohmic, resistive and uniform medium:
    \begin{equation}
      \sigma_e \nabla^2 \Phi = -I_m.
    \end{equation}
    Here $\sigma_e$ is the electric conductivity of the extracellular medium
    and $I_m$ is the density of current sources. A more general approach
    can be seen at \cite{Bedard11}.
    
  \end{frame}


  \begin{frame}
     \frametitle{Implementation}
     The more direct approach for the CSD in 2D would be to apply
     a discrete Laplacian Filter directly on the data, if
     the spacing of the data were of the order of the dipolar structures
     that we wish to see. Our main problem is that such a filter is
     \emph{extremely} sensitive to noisse. This noice can be clearly seen
     on the temporal domain as random ls spikes ranging  $-40\mu V$ to 
     $+40\mu V$ 
  \end{frame}
% etc
\end{document}

