\documentclass[xetex,mathserif,serif]{beamer}

%\usepackage[numberedbib]{apacite}
%\usepackage[english]{babel}
\usepackage{natbib}
\usepackage{bibentry}
\bibliographystyle{apalike}


\begin{document}
  \begin{frame}
    \frametitle{CSDA}
    Our measurements would allow us to obtain the \emph{Current Source Density}
    of the electric field potentials  recorded in the experiments.
    The fundamental equation which relates this two quantities is
    Gauss Law for an ohmic, resistive and uniform medium:
    \begin{equation}
      \sigma_e \nabla^2 \Phi = -I_m.
    \end{equation}
    Here $\sigma_e$ is the electric conductivity of the extracellular medium
    and $I_m$ is the density of current sources. A deeper approach
    can be seen at \cite{Bedard11}.
    
  \end{frame}

   \begin{frame}
     \frametitle{Implementation: de-noising the Data}
     A direct application of a Laplacian Filter is not viable, since
     this operator is extremely sensitive to noise. The noise is coupled
     to the line power (\~ 60Hz). A (temporal) Gaussian convolution takes care
     of this and its harmonics.
  \end{frame}
  % etc


  \begin{frame}
     \frametitle{Implementation: de-noising the Data}
     The Laplacian implementation is, in this discrete form, a convolution
     with the following matrix. This is in accordance with the
     scales that we are interested in. Compare with \cite{Vaknin88}.
    
  \end{frame}


  \begin{frame}
     \frametitle{Enter of Mass Dynamics}
     Following \cite{Manjarrez07}, we desire to follow a putative CoM,
     to discover how activity propagates along the CA.
     
  \end{frame}


  

  \begin{frame}
     \frametitle{Implementation: naïve search}
     A direct approach proves useless, as the activity occurs at separated
     sites along CA1 and CA3. The concave geometry of our slice
     produces a centre of mass outside the CA, and without physiological
     interpretation.
  \end{frame}


  
  \begin{frame}
     \frametitle{Implementation: disjoint set detection.}
     It is better to focus on the \emph{disjoint sets} revealed by
     the CSDA. We have a clear separation between sources and sinks,
     and every set would have a centre of mass inside the CA, thus
     allowing us to follow the propagation of activity on different
     components \emph{simultaneously}.
     
  \end{frame}

  
  \begin{frame}{Bibliography}
   % \bibliographystyle{apacite}
    \bibliography{BiblioReportes01}
  \end{frame}


  
  
\end{document}

