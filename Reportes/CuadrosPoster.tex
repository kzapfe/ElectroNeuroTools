\documentclass{article}

\usepackage[utf8]{inputenc}


\begin{document}

\section{CSD and centers of mass}


For certain purposes the representation of the measured data in CSD
representation has advantages over the recorded LFP. The principal one is
that it minimizes the volume-conducted activity, pointing more acurattely
to the sites in which electrical-measurable events are taking place. We have
another more important reason to resort to this representation: namely,
as it names implies, it represents a \emph{density}, so, it indicates
quantity of something per unit volume, which permits us to apply certain
concepts in a rigourous, physicaly interpretable manner.

Densities are instrumental for taking vector averages, also known as centers
of mass. A charged or flux density is in this regard no different that a
mass density and can be used to obtain significant ``centers'' of
physical objects. In our case, our physical objects are active neurons and
their inmediate extracellular space, and is our interest to track a
putative ``center of activity'' in order to trace routes in which this
activity flows across the tissues that we are studing. Also, in this regard,
it delimitates (instantaneous) active areas from their surroundings by
a (aproximate) zero Set, which is the locus of all points in the measurement
area that register a null CSD. This set is typically a regular set of
curves which separate two sets, the ``sources'' and ``sinks'', i.e. the
set of all places where there is outflux of charge and the set of all places
where there is influx of charge. Each of these sets can be decomposed in
its disjoint components using the zero set as the border between them.

Each disjoint component is an instantaneous locus where we localize one pole of
an active unit. These poles will appear to move as the activity grows in neighbouring
units or vanishes. The high frecuency of sampling at our disposal permits us to
track very precicely the apparent track that this apparition, grows, wanning and
dissapearence of poles produce. Thus, by calculating the center of mass of
each one of these poles at succesive recording times
we can obtain a trayectory of activity at local units. 


The HD MEAS of the kind of the BioCAM 4096  an electrode density comparable
to the sizes of of typical neurons at the hipocampal tissue, which is around
the limit in which mean size approximations are still applicable \cite{Bedard2011}.
At such ranges a simple convex numeric Laplacian yields adecuate results without
the need to restore to computer costly alternatives such as the iCSDA or kCSDA
\cite{Leski2011, Potworowsky2011}. A nice implementation of a 2D numeric Laplacian
which reduces the typical cross effect is given by a convolution of
the 2D array of data with the following matrix:
\begin{equation}

\end{equation}






\end{document}

