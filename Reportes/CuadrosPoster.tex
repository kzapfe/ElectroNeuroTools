\documentclass{article}

\usepackage[utf8]{inputenc}


\begin{document}

\section{CSD and centers of mass}


For certain purposes the representation of the measured data in CSD
representation has advantages over the recorded LFP. One is
that it minimizes the volume-conducted activity, pointing more acurattely
to the sites in which electrical-measurable events are taking place \cite{Mitzdorf85}.
We have another important reason to resort to this representation:
as it names implies, it represents a \emph{density}, so, it indicates
quantity of something per unit volume. Thus we are able to apply certain
concepts in a rigourous, physicaly interpretable manner.

Densities are instrumental for taking vector averages, also known as centers
of mass. A charge or flux density is in this regard no different that a
mass density and can be used to obtain significant ``centers'' of
physical quantities. In our case, this are the ionic flows across
the membrane, which, looked macroscopically from the extracellular
space, appear as sinks and sources. These are evidence of neuronal
activity and is and is our interest to locate a
putative ``instantaneous center'' of said activity and then track
trajectories which could sintetize the sequence of diverse units
incorporating themselves to the action.

Sinks and sources are separated by a zero set. This is also an advantage
over the LFP representation, in which any particular value has no
interpretative significance, and only \emph{differences} between values
have meaning. 
Our zero  set is typically a union of regular 
curves which separate sources from sinks. As we choose a numerical error
tolerance, these sets appear in our measurements conformed by disjoint
components. This permits us two confirm measurements that reveal
the expected polar activity of firing neurons, and to locate centers
of  activity by calculating the center of mass in each disjoint component
at each frame from our data. We then have an instantaneous ``center of mass''
for each disjoint component, which we shall call it the ``instantaneous local
center of activity'' or ILCA for short. These ILCAs have an apparent
movement as sinks and sources appear and fade, so we can track their
position and intensity frame by frame, producing a putative trayectory,
which could reveal displacement of activity or the succesion of recruiting
neurons. 
Thanks to the high frecuency sampling of our experiments, we can observe
very small displacements of this centers of activities. Care must be taken
to interpret this displacements below the electrode scale.

Each trayectory thus obtained encodes and synthetizes the following information:
\begin{description}
\item[Start] location and time of the apparition of the source/sink.
\item[Path] succesive locus and timestamp of the putative centers.
\item[Intensity] The time dependant integrated density of the source/sink.  
\item[End] location and time of the disapearence of the source/sink.
\end{description}

Visual comparition with the sink/source spatiotemporal structure coincides
with known measuraments of CSD. 


Each disjoint component is an instantaneous locus where we localize one pole of
an active unit. These poles will appear to move as the activity grows in neighbouring
units or vanishes. The high frecuency of sampling at our disposal permits us to
track very precicely the apparent track that this apparition, grows, wanning and
dissapearence of poles produce. Thus, by calculating the center of mass of
each one of these poles at succesive recording times
we can obtain a trayectory of activity at local units. 


The HD MEAS of the kind of the BioCAM 4096  an electrode density comparable
to the sizes of of typical neurons at the hipocampal tissue, which is around
the limit in which mean size approximations are still applicable \cite{Bedard2011}.
At such ranges a simple convex numeric Laplacian yields adecuate results without
the need to restore to computer costly alternatives such as the iCSDA or kCSDA
\cite{Leski2011, Potworowsky2011}. A nice implementation of a 2D numeric Laplacian
which reduces the typical cross effect is given by a convolution of
the 2D array of data with the following matrix:
\begin{equation}
f(x)
\end{equation}






\end{document}

