\documentclass{article}

%\usepackage[utf8]{inputenc}
\usepackage{fontspec}
\usepackage{amsmath, amssymb}

\usepackage{graphicx}

\usepackage{caption}
\usepackage{subcaption}



\setmainfont[Ligatures=TeX]{Oldstyle HPLHS}

\newcommand{\Jd}{\mathbf{J}}
\newcommand{\EF}{\mathbf{E}}
\newcommand{\cond}{\boldsymbol{\sigma}}
\DeclareMathOperator{\diver}{div}
\DeclareMathOperator{\grad}{grad}

\title{Two Dimensional CSDA at the scale of neuronal tissue.}
\author{W. P. K. Zapfe, Whoever Helps}

\begin{document}

\maketitle

\begin{abstract}
As high density electrode arrays become more common 
electrophysiological measurements of activity in neural tissue approach
the scale of individual neurons. The electrodes measure
directly the Local Field Potential, but it is aknowledged that the
Current Source Density has more resolving power and distinguishes
clearly between the loci of outgoing and ingoing currents.
 The role of the conductivity tensor is rarely taken into account
in the derivation of the CSD, and this could potentialy lead to
spurious distributions of the sources and sinks. 
In dense neural tissue, such as the
stratum piramidale layer of the hippocampus, clear
deviations from average conductivity have been measured. 
Here we show a possible qualitative treatment of the conductivity
across the layers of the hippocampus based on experimental evidence
and general smoothness arguments.
\end{abstract}


\section{Introduction}

As Bédard and Destexhe pointed out in \cite{Bedard11}, the CSDA, as it
is performed usually, rests on
assumptions which could leave important features out. One of these 
is that the extracellular medium has very homogeneous 
electrical propierties. This may depend on the scale in which measurements
take place and the size of the neuronal structures involved. We shall
focus our atention on the hippocampus of the rat and mouse
 and its typical sizes in what follows. 

The CSDA is the application of charge conservation theorem
for ionic dilutions in the extracelular medium for neuronal
tissue. CSD are obtained either directly through application
of a numeric Laplacian operator directly to LFP data \cite{Nicholson75}, 
or through  more refinated
techniques, called inverse CSDA \cite{Pettersen2006, Leski2011} or kernel CSDA 
\cite{Potworowsky2011}. There the conductivity $\cond$ is taken as 
 a numerical constant, that is:
\begin{equation}\label{CSDasLaplacian}
-\nabla^2 \Phi = \frac{I_m}{\cond}. 
\end{equation}
Here $\Phi$ is the LFP and $I_m$ the current source density. 
This is indeed a reasonable treatment at scales larger than $100\mu m$,
where each electrode measures averaged contributions from many different
densities of activity and structures. Even more, data are 
generaly smoothed out for presentation, and this contributes to the
averaging process.

On the other hand, it has been shown that at 
dense neuronal layers such as the piramidal
strata in the hippocampus the conductivity (or equivalently,
the resistivity) varies greatly around its mean value along
an axis which transverses said structure
\cite{Holsheimer87, Lopez01, TrevinoPersonal}. 
The evidence points out that resistivity grows 
considerably around the thick somatic layers. These layers 
have thickness between $40\mu m$ to $100 \mu m$, which is
near or above the interelectrode spacing of stat-of-the-art MEAS, so
differences in conductivity shall be noticiable at this
level of resolution. In the following we shall analytical address this
issue, then, we will implement a model ``conductivity variation'' 
and use it on real LFP data, compare
the results with those done without this factor and then conclude.


\section{Analytical  derivation of the CSD}

The following linear relationship is a good aproximation for
most media at convencional regimes: 
\begin{equation}
\Jd=\cond \EF,
\end{equation}
that is, the current density $\Jd$ is the result of the
local electric field $\EF$ and the local conductivity, $\cond$.
We usually take this quantity as constant, 
at least over macroscopic scales.
The conductivity in inhomogeneous media can be rarely
simplified this much, as it can depend both from the direction
of the fields and currents, and vary from point to point. 
Then the above expression is just shorthand notation for the
explicit  \emph{tensorial} relationship:
\begin{equation}
  \begin{pmatrix}
    J_x \\
    J_y \\
    J_z
  \end{pmatrix}
  =
   \begin{pmatrix}
     \sigma_{xx}E_x+ \sigma_{xy}E_y+\sigma_{xz}E_z \\
     \sigma_{xy}E_x+ \sigma_{yy}E_y+\sigma_{yz}E_z \\
     \sigma_{xz}E_x+ \sigma_{yz}E_y+\sigma_{zz}E_z \\
  \end{pmatrix},
\end{equation}
where we allready used the fact that $\cond$ is simmetric when
expressed on Cartessian coordinates. All quantities are functions of
both position and time.
The electric field is in the quasi-static approximation the
gradient of the LFP, $\Phi$:
\begin{equation}
  -\nabla \Phi=\EF
\end{equation}
To obtain the sources and sink densities, we use charge conservation in
its differential form:
\begin{equation}
\nabla \cdot \Jd +\frac{\partial \rho}{\partial t}=0
\end{equation}
The usual definition of $-\partial_t \rho =I_m$ will be used here. 
We perform the divergence operation on the current density, \emph{not}
on the electric field. When the medium is isotropic and homogeneous
this derivation coincides within a positive factor with the one
on eq. \ref{CSDasLaplacian}.
This makes sense when we are dealing with macroscopic quantities,
that is, that they are at last one order of magnitude larger
than the underlying physical structures which could cause 
inhomogenities, but as measurements devices get finner,
this treatment begins to be faulty. 
Our primary concern are the roles of the thick somatic layers of
the hippocampus of laboratory rodents and the implications for
CSDA at scales of $20 \thicksim 50 \mu m$. 


In experiments of this sort performed on hippocampus slices, the
MEAs permit us to measure a sample of LFP over discrete points on a
two dimensional lattice, so we do not have possibility of
calculating currents or electric fields on the third dimension of space.
The slices of the hippocampus are very thin, and we do not consider
the posibility of current going outside the liquid media in which
the sample lies (in one
direction there is air and on the other the MEA), so we
can reduce our formalism to two dimensional space. We then take
$J_z=0$ and use $(x,y)$ as a cartesian coordinate system aligned
with the MEA. Then the \emph{flat} current density  is:
\begin{equation}
  \begin{pmatrix}
    J_x\\
    J_y
  \end{pmatrix}  =
  \begin{pmatrix}
    \sigma_{xx}E_x+\sigma_{xy}E_y \\
    \sigma_{xy}E_x+\sigma_{yy}E_y 
  \end{pmatrix}=
  \begin{pmatrix}
    -\sigma_{xx}\partial_x \phi -\sigma_{xy}\partial_y \phi \\
    -\sigma_{xy}\partial_x \phi -\sigma_{yy}\partial_y \phi 
  \end{pmatrix}.
\end{equation}
The divergence, in cartesian coordinates, yields:
\begin{equation}\label{Imcart}
  \begin{split}
  -I_m= & (\partial_x \sigma_{xx}) (\partial_x \phi) +
  (\partial_y \sigma_{yy}) (\partial_y \phi)\\
& +(\partial_x \sigma_{xy}) (\partial_y \phi) +(\partial_y \sigma_{xy}) (\partial_x \phi)\\
& + 2\sigma_{xy} \partial_x\partial_y \phi \\
& + \sigma_{xx} \partial_x^2 \phi + \sigma_{yy} \partial_y^2 \phi.
\end{split}
 \end{equation}
The \emph{tacit} assumption that can be noted in most of the CSDA previously
done is that the derivatives of $\cond$ are negligible and only
the terms on the last line are  taken into account. 
It can be argued that for scales that
encompas many neurons and interneuronal space, the averaged conductivity is
constant and has then its derivatives vanish. For MEAS that take measuraments
more than $100 \mu m$ appart, this may be a reasonable assumption (see
for example, eq. 12 in \cite{Bedard11} and the arguments leading to it).
Examples of such derivations are those obtained by the MED64 probe in the
work of Wang et all \cite{Wang10}, etc etc. Also sofisticated
numerical methods, such as iCSD \cite{Pettersen2006, Leski11}
and kCSD \cite{Potworowsky2011} use this approximation, albeit
explicitly. 

If the interelectrode spacing
is less than $100 \mu m$ then the neuronal tissue would begin
to play a role in the inhomogenities of $\cond$, although the isotropic
assumption may still be valid at scales larger than $20 \mu m$. At such 
scales the laminar structure of the Cornu Amonis or the Gyrus Dentate is
the main feature, and we should expect variations on the electrical 
propierties of the media along the apical direction. Measurements made
by Holsheimer showed that measurements in the CA1 region of a mouse
hipocampus show variations of  $40\% $ in the resistivity around the mean
value at distances of about $60 \mu m$.
\cite{Holsheimer87}. This measurements have been confirmed
recently by a collaborator \cite{TrevinoPersonal}. On the other hand
we expect very little variation of the conductivity tensor along
each structural layer. The symmetry of the conductivity tensor represents
the structural underlying propierties of the media: as it can be 
represented on a diagonal form on a suitable set of \emph{orthogonal}
coordinates, then such coordinates must be associated with the 
structures that give $\cond$ its values. Then the representation
of $\cond$ in diagonal form should represent the conductivity
along the axonal-apical direction, with high variation, and another
axis transverse to it, in which $\cond$ showing very little variation.
We can imagine that we can set a locally orthonormal set of coordinates
alignate with such structures (see fig. \ref{esquemas01}).
 Then the expression in eq. \ref{Imcart}
can be reduced substantialy.  The $\cond$ tensor would appear as
\begin{equation}
\tilde{\cond}=
\begin{pmatrix}
\sigma_t & 0 \\
0 & \sigma_a
\end{pmatrix}
\end{equation} 
where $(t,a)$ are the transveral and axonal-apical local coordinates.
The CSD would be obtained as
\begin{equation}\label{Imapic}
\begin{split}
-I_m= & (\partial_t \sigma_{t}) (\partial_t \phi) +
(\partial_a \sigma_{a}) (\partial_a \phi)  \\
& + \sigma_{a} \partial_a^2 \phi + \sigma_{t} \partial_t^2 \phi. 
\end{split}
\end{equation}

   

\begin{figure}[h]
\centering
\begin{subfigure}[t]{0.20\textwidth}
\includegraphics[width=\textwidth]{DiagramaCAetDG02.pdf}
\caption{}
\label{diagCA}
\end{subfigure}
\quad
\begin{subfigure}[t]{0.33\textwidth}
\includegraphics[width=0.67\textwidth]{DiferentialCoordinates01.pdf}
\caption{}
\label{loceig}
\end{subfigure}
\begin{subfigure}[t]{0.20\textwidth}
\includegraphics[width=\textwidth]{PseudoCoordinates01.pdf}
\caption{}
\label{pseudocor}
\end{subfigure}
\caption{\ref{diagCA}) A diagram showing the position of
the hippocampus slice over the MEA. \ref{loceig}) 
A tangent space basis along the
eigenvectors of $\cond$. The red vectors indicate the
Cartesian Coordinates aligned with the MEA latice. 
\ref{pseudocor}) A putative local coordinate system spawned by the 
eigenvectors $(t,a)$ at every point of intereset. 
 }\label{diagdif}
\label{esquemas01}
\end{figure}



If the medium where the ions diffuse is liquid, and the structures
therein are randomly distributed, the assumption of isotropy may hold 
on average in sufficiently big scales \cite{Bedard11}. 
In order to not make the notation more cumbersome, let us aggree
that we are dealing with averages over volumes of the aproximate size
of the electrodes. Then, 
$\sigma_t=\sigma_a=\sigma$. As we stated, there is no
change along the strata, but there is change along the appical
direction:
\begin{align}
\partial_t \sigma & =0 \\
\partial_a \sigma & \neq 0. 
\end{align}
We are then left with
\begin{equation}
I_m=-\partial_t \sigma \partial_t \phi - \sigma \nabla^2 \phi.
\end{equation}
Here we must point out that the Laplacian is an invariant operator,
so the last summand of the equation could be in principle be implemented
in \emph{any} convenient set of coordinates, but in numerical aplications,
the \emph{Discrete Laplacian Operator} is \emph{not} invariant (further
discution in the following section). 
The term $\partial_t \sigma \partial_t \phi$ is the hindrance to 
ionic displacement due to dense layers of soma and dendrites near the
CA structure. Most autors have neglected its efects, and in one
dimensional analysis the structure of sources and sinks is not affected
by this. Now, as the aquisition of data in two dimensions becomes
finer, this term could lead to inexactitude in the location of the
sinks and sources, depending on the magnitude of the
derivative of the conductance. Locally both $(x,y)$ and
$(t,a)$ are orthonormal systems, so we can change from one to another 
with a rotation. If $theta$ is the angle between the $x$ vector
and $t$, the partial derivatives follow the next set of relations:
\begin{align}
\partial_t \sigma &= \cos\theta \partial_x \sigma +\sin\theta \partial_y \sigma=0\\
\partial_a \sigma &= -\sin\theta \partial_x \sigma +\cos\theta \partial_y \sigma
\end{align}

After a bit of algebraic juggling we end up with
\begin{equation}
-I_m=\sigma\nabla^2 \phi + 
\partial_x\sigma (\partial_x \phi - \tan \theta \partial_y \phi)
\end{equation} 


In order to estimate the influence
of this summand on the result, we shall make some toy models which
represent the variation of $\sigma$ along the apical direction.






For a carefull and complete derivation and discution of the CSD, 
see \cite{Bedard11}. We start with the charge conservation equation,
\begin{equation}
\nabla \cdot \Jd + \frac{d \rho}{d t} =0.
\end{equation}



\bibliographystyle{plain}
\bibliography{BiblioReportes01}



\end{document}
