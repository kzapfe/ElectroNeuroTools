\documentclass{article}

%\usepackage[utf8]{inputenc}
\usepackage{fontspec}
\usepackage{amsmath}

\usepackage{graphicx}

\usepackage{caption}
\usepackage{subcaption}



\setmainfont[Ligatures=TeX]{Oldstyle HPLHS}

\newcommand{\Jd}{\mathbf{J}}
\newcommand{\EF}{\mathbf{E}}
\newcommand{\cond}{\boldsymbol{\sigma}}
\DeclareMathOperator{\diver}{div}
\DeclareMathOperator{\grad}{grad}

\title{Two Dimensional CSDA at the scale of neuronal tissue.}
\author{W. P. K. Zapfe, Whoever Helps}

\begin{document}

\maketitle

\begin{abstract}
As high density electrode arrays become cheaper and more common, 
electrophysiological measurements of activity in neural tissue approach
the scale of the underlying structures. The electrodes measure
directly the Local Field Potential, but it is aknowledged that the
Current Source Density has more resolving power and distinguishes
clearly between the loci of outgoing current and ingoing currents.
 The role of the conductivity tensor is rarely taken into account
in the derivation of the CSD, and this could potentialy lead to
spurious distributions of the sources and sinks- 
In dense neural tissue, such as the
stratum piramidale layer of the hippocampus, clear
deviations from average conductivity have been measured. 
Here we show a possible qualitative treatment of the conductivity
across the layers of the hippocampus, and compare the results
when this inhomogenity is taken into account and when its not.
\end{abstract}


\section{Introduction}

As Bédard and Destexhe pointed out in \cite{Bedard11}, the CSDA, as it
is performed usually, rests on
assumptions which could leave important features out. One of these 
assumptions is that the extracellular medium has very homogeneous 
electrical propierties. It has been shown not to be
the case for dense neuronal layers such as the piramidal
strata in the hippocampus \cite{Holsheimer87, Lopez01, TrevinoPersonal}.
The evidence mostly points out that resistivity grows 
considerably around thick somatic layers.
The following linear relationship is a good aproximation for
most media at convencional regimes: 
\begin{equation}
\Jd=\cond \EF,
\end{equation}
that is, the current density $\Jd$ is the result of the
local electric field $\EF$ and the local conductivity, $\cond$.
We usually take this quantity as constant, 
at least over macroscopic scales.
The conductivity in inhomogeneous media can be rarely
simplified this much, as it can depend both from the direction
of the fields and currents, and vary from point to point. 
Then the above expression is just shorthand notation for the
explicit  \emph{tensorial} relationship:
\begin{equation}
  \begin{pmatrix}
    J_x \\
    J_y \\
    J_z
  \end{pmatrix}
  =
   \begin{pmatrix}
     \sigma_{xx}E_x+ \sigma_{xy}E_y+\sigma_{xz}E_z \\
     \sigma_{xy}E_x+ \sigma_{yy}E_y+\sigma_{yz}E_z \\
     \sigma_{xz}E_x+ \sigma_{yz}E_y+\sigma_{zz}E_z \\
  \end{pmatrix},
\end{equation}
where we allready used the fact that $\cond$ is simmetric when
expressed on Cartessian coordinates.
We assume that the electric field can be obtained as the
gradient of the LFP, $\phi$:
\begin{equation}
  -\nabla \phi=\EF
\end{equation}
This LFP is the quantity that the microelectrode arrays (MEAs) permits us
to measure on the extracelular medium.
To obtain the sources and sink densities, we use Gauss Divergence Theorem,
\begin{equation}
\nabla \cdot \Jd =I_m
\end{equation}
The usual definition of $-\partial_t \rho =I_m$ has been made here. 
We perform the divergence operation on the current density, \emph{not}
on the electric field. When the medium is isotropic and homogeneous
both derivations coincide within a positive factor.
This makes sense when we are dealing with macroscopic quantities,
that is, that the are at last one order of magnitude larger
than the underlying physical structures which could cause 
inhomogenities, but as our measurements get finner and finner,
this treatment begins to be faulty. 
In the next section we investigate the role of the inhomogenities 
of the conductivity on current density
($\Jd$) and current source density ($I_m$) at scales around
the $20\mu m$. 


\section{Analytical  derivation of the CSD}


In experiments of this sort performed on hippocampus slices, the
MEAs permit us to measure a sample of LFP over discrete points on a
two dimensional lattice, so we do not have possibility of
calculating currents or electric fields on the third dimension of space.
The slices of the hippocampus are very thin, and we do not consider
the posibility of current going outside the neural tissue sample (in one
direction there is air and on the other the MEA), so we
can reduce our formalism to two dimensional space. We then take
$J_z=0$ and use $(x,y)$ as a cartesian coordinate system aligned
with the MEA. Then the \emph{flat} current density  is:
\begin{equation}
  \begin{pmatrix}
    J_x\\
    J_y
  \end{pmatrix}  =
  \begin{pmatrix}
    \sigma_{xx}E_x+\sigma_{xy}E_y \\
    \sigma_{xy}E_x+\sigma_{yy}E_y 
  \end{pmatrix}=
  \begin{pmatrix}
    -\sigma_{xx}\partial_x \phi -\sigma_{xy}\partial_y \phi \\
    -\sigma_{xy}\partial_x \phi -\sigma_{yy}\partial_y \phi 
  \end{pmatrix}.
\end{equation}
The divergence, in cartesian coordinates, yields:
\begin{equation}\label{Imcart}
  \begin{split}
  -I_m= & (\partial_x \sigma_{xx}) (\partial_x \phi) +
  (\partial_y \sigma_{yy}) (\partial_y \phi)\\
& +(\partial_x \sigma_{xy}) (\partial_y \phi) +(\partial_y \sigma_{xy}) (\partial_x \phi)\\
& + 2\sigma_{xy} \partial_x\partial_y \phi \\
& + \sigma_{xx} \partial_x^2 \phi + \sigma_{yy} \partial_y^2 \phi.
\end{split}
 \end{equation}
The \emph{tacit} assumption that can be noted in most of the CSDA previously
done is that the derivatives of $\cond$ are negligible and only
the terms on the last line are  taken into account. 
It can be argued that for scales that
encompas many neurons and interneuronal space, the averaged conductivity is
constant and has then its derivatives vanish. For MEAS that take measuraments
more than $100 \mu m$ appart, this may be a reasonable assumption (see
for example, eq. 12 in \cite{Bedard11} and the arguments leading to it).
Examples of such derivations are those obtained by the MED64 probe in the
work of Wang et all \cite{Wang10}, etc etc. Also sofisticated
numerical methods, such as iCSD \cite{Pettersen2006, Leski11}
and kCSD \cite{Potworowsky2011} use this approximation, albeit
explicitly. 

If the interelectrode spacing
is less than $100 \mu m$ then the neuronal tissue would begin
to play a role in the inhomogenities of $\cond$, although the isotropic
assumption may still be valid at scales larger than $20 \mu m$. At such 
scales the laminar structure of the Cornu Amonis or the Gyrus Dentate is
the main feature, and we should expect variations on the electrical 
propierties of the media along the apical direction. Measurements made
by Holsheimer showed that measurements in the CA1 region of a mouse
hipocampus show variations of  $40\% $ in the resistivity around the mean
value at distances of about $60 \mu m$.
\cite{Holsheimer87}. This measurements have been confirmed
recently by a collaborator \cite{TrevinoPersonal}. On the other hand
we expect very little variation of the conductivity tensor along
each structural layer. The symmetry of the conductivity tensor represents
the structural underlying propierties of the media: as it can be 
represented on a diagonal form on a suitable set of \emph{orthogonal}
coordinates, then such coordinates must be associated with the 
structures that give $\cond$ its values. Then the representation
of $\cond$ in diagonal form should represent the conductivity
along the axonal-apical direction, with high variation, and another
axis transverse to it, in which $\cond$ showing very little variation.
We can imagine that we can set a locally orthonormal set of coordinates
alignate with such structures (see fig. \ref{esquemas01}).
 Then the expression in eq. \ref{Imcart}
can be reduced substantialy.  The $\cond$ tensor would appear as
\begin{equation}
\tilde{\cond}=
\begin{pmatrix}
\sigma_t & 0 \\
0 & \sigma_a
\end{pmatrix}
\end{equation} 
where $(t,a)$ are the transveral and axonal-apical local coordinates.
The CSD would be obtained as
\begin{equation}\label{Imapic}
\begin{split}
-I_m= & (\partial_t \sigma_{t}) (\partial_t \phi) +
(\partial_a \sigma_{a}) (\partial_a \phi)  \\
& + \sigma_{a} \partial_a^2 \phi + \sigma_{t} \partial_t^2 \phi. 
\end{split}
\end{equation}

   

\begin{figure}[h]
\centering
\begin{subfigure}[t]{0.20\textwidth}
\includegraphics[width=\textwidth]{DiagramaCAetDG02.pdf}
\caption{}
\label{diagCA}
\end{subfigure}
\quad
\begin{subfigure}[t]{0.33\textwidth}
\includegraphics[width=0.67\textwidth]{DiferentialCoordinates01.pdf}
\caption{}
\label{loceig}
\end{subfigure}
\begin{subfigure}[t]{0.20\textwidth}
\includegraphics[width=\textwidth]{PseudoCoordinates01.pdf}
\caption{}
\label{pseudocor}
\end{subfigure}
\caption{\ref{diagCA}) A diagram showing the position of
the hippocampus slice over the MEA. \ref{loceig}) 
A tangent space basis along the
eigenvectors of $\cond$. The red vectors indicate the
Cartesian Coordinates aligned with the MEA latice. 
\ref{pseudocor}) A putative local coordinate system spawned by the 
eigenvectors $(t,a)$ at every point of intereset. 
 }\label{diagdif}
\label{esquemas01}
\end{figure}



If the medium where the ions diffuse is liquid, and the structures
therein are randomly distributed, the assumption of isotropy may hold 
on average in sufficiently big scales \cite{Bedard11}. 
In order to not make the notation more cumbersome, let us aggree
that we are dealing with averages over volumes of the aproximate size
of the electrodes. Then, 
$\sigma_t=\sigma_a=\sigma$. As we stated, there is no
change along the strata, but there is change along the appical
direction:
\begin{align}
\partial_t \sigma & =0 \\
\partial_a \sigma & \neq 0. 
\end{align}
We are then left with
\begin{equation}
I_m=-\partial_t \sigma \partial_t \phi - \sigma \nabla^2 \phi.
\end{equation}
Here we must point out that the Laplacian is an invariant operator,
so the last summand of the equation could be in principle be implemented
in \emph{any} convenient set of coordinates, but in numerical aplications,
the \emph{Discrete Laplacian Operator} is \emph{not} invariant (further
discution in the following section). 
The term $\partial_t \sigma \partial_t \phi$ is the hindrance to 
ionic displacement due to dense layers of soma and dendrites near the
CA structure. Most autors have neglected its efects, and in one
dimensional analysis the structure of sources and sinks is not affected
by this. Now, as the aquisition of data in two dimensions becomes
finer, this term could lead to inexactitude in the location of the
sinks and sources, depending on the magnitude of the
derivative of the conductance. Locally both $(x,y)$ and
$(t,a)$ are orthonormal systems, so we can change from one to another 
with a rotation. If $theta$ is the angle between the $x$ vector
and $t$, the partial derivatives follow the next set of relations:
\begin{align}
\partial_t \sigma &= \cos\theta \partial_x \sigma +\sin\theta \partial_y \sigma=0\\
\partial_a \sigma &= -\sin\theta \partial_x \sigma +\cos\theta \partial_y \sigma
\end{align}

After a bit of algebraic juggling we end up with
\begin{equation}
-I_m=\sigma\nabla^2 \phi + 
\partial_x\sigma (\partial_x \phi - \tan \theta \partial_y \phi)
\end{equation} 


In order to estimate the influence
of this summand on the result, we shall make some toy models which
represent the variation of $\sigma$ along the apical direction.






For a carefull and complete derivation and discution of the CSD, 
see \cite{Bedard11}. We start with the charge conservation equation,
\begin{equation}
\nabla \cdot \Jd + \frac{d \rho}{d t} =0.
\end{equation}



\bibliographystyle{plain}
\bibliography{BiblioReportes01}



\end{document}
