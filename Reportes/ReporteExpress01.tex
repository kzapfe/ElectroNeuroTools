\documentclass{beamer}


\usepackage{amsmath}
\usepackage[utf8]{inputenc}
\usepackage[spanish]{babel}
\usepackage{color}

\decimalpoint

\usefonttheme{serif}

\newcommand{\dd}{\, \mathrm{d}}
\newcommand{\xq}{\, \mathbf{x}}

\begin{document}


\begin{frame}
  \frametitle{De la actividad evocada a las trayectorias de los CM}
  Dado que la actividad evocada produce respuestas
  \emph{perfectamente} acompazadas, podemos
  promediar sobre varios eventos experimentales
  a fin de tener una señal más limpia. 
  \begin{center}
   \includegraphics[width=0.8\textwidth]{PotencialesEvocados01.pdf} 
  \end{center}
\end{frame}


\begin{frame}
  \frametitle{¿Se vale promediar sobre CSD?}
    Téoricamente sì: el CSDA es una operación lineal
    (formas más sofisticadas pueden no serlo). Esto es:
    
\begin{equation}
 \langle CSD(\phi) \rangle = CSD (\langle \phi \rangle)
\end{equation}

\begin{center}
  \begin{tabular}{cc}
    \includegraphics[width=0.5\textwidth]{CSDEvocado-Trancazo_1-0177.png} &
    \includegraphics[width=0.5\textwidth]{CSDEvocado-Promedio-0177.png} 
  \end{tabular}
\end{center}
    
\end{frame}


\begin{frame}
  \frametitle{No podemos \emph{``cerar''} los canales saturados}
  Un criterio simple nos dice cuando un canal no registro nada.

  El ataque usual era poner ese canal en ceros, pero al derivar
  nos produce un efecto de borde muy grande.
  
\begin{center}
  \begin{tabular}{cc}
    \includegraphics[width=0.4\textwidth]{LFPEvocado_1R4-Promedio-177.png} &
    \includegraphics[width=0.4\textwidth]{CSDEvocado-Promedio-0177.png} 
  \end{tabular}
\end{center}
Solución: tenemos que hacer un promedio sobre la vecindad local
para cada canal saturado, en cada cuadro.
\end{frame}


\begin{frame}
\frametitle{¿Promedio de Trayectorias?}
Usando normalización sobre la longitud de arco, el
promedio de varias curvas esta bien definido.
\begin{center}
\includegraphics[width=0.5\textwidth]{PromedioSobreUnaTrayectoria01.pdf}
\end{center}
\end{frame}


\begin{frame}
  \frametitle{¿Promedio sobre los Centros de Masa?}
  Si el centro de masa fuera sobre todo el espacio, una
  vez más sería válido por linealidad. ¡La
  detección de componentes disjuntos nos puede echar todo a
  perder!
\begin{center}
 \includegraphics[width=0.5\textwidth]{promedioComponentesDisjuntos01.pdf}
\end{center}
\end{frame}


\begin{frame}
\frametitle{Ceros de funciones similares}
  \frametitle{Posible remedio:}
  Si tenemos una familia de funciones \emph{suficientemente cercanas}
  podemos garantizar
  que sus ceros son deformaciones pequeñas unos de otros. 
\begin{center}
 \includegraphics[width=0.7\textwidth]{FuncionesRegularesCeros01.png}
\end{center}
\end{frame}


\begin{frame}
  \frametitle{Funciones similares}
  La actividad evocada (ya sea LFP o CSD) tal vez puede ser descompuesta
  en una parte estrictamente determinista y una parte \emph{ruidosa}
  o \emph{espontanea}. La parte determinista determinará las propiedas globales de la
  función y será cercana al promedio.
\begin{center}
 \includegraphics[width=0.9\textwidth]{FuncionesRegulares01.png}
\end{center}
\end{frame}


\begin{frame}
  \frametitle{Funciones similares}
  La actividad evocada (ya sea LFP o CSD) tal vez puede ser descompuesta
  en una parte estrictamente determinista y una parte \emph{ruidosa}
  o \emph{espontanea}. La parte determinista determinará las propiedas globales de la
  función y será cercana al promedio.
\begin{center}
 \includegraphics[width=0.9\textwidth]{FuncionesRegulares01.png}
\end{center}
\end{frame}


\begin{frame}
  \frametitle{Inspección visual}
\begin{center}
  \includegraphics[width=0.93\textwidth]{NivelesEncimados3experimentos01.png} 
 % \includegraphics[width=0.43\textwidth]{NivelesEncimados-3-01.png} 
\end{center}
  %\includegraphics[width=0.33\textwidth]{NivelesEncimados-4-01.png} 

\end{frame}



\end{document}
