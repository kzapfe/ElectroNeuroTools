\documentclass{beamer}


\usepackage{amsmath}
\usepackage[utf8]{inputenc}
\usepackage[spanish]{babel}
\usefonttheme{serif}

\begin{document}


\begin{frame}
  \frametitle{Medidas y Rastreo de Actividad Neuronal}
 ¿Podemos obtener un \emph{rastreo de la actividad}?
 % Contamos con un aparato que nos permite medir de forma densa espacial y temporalmente
 % los potencials de campo de tejido neuronal.
\center
  \begin{tabular}{cc}
    \includegraphics[width=0.4\textwidth]{LFP-cb-716.png} &
    \includegraphics[width=0.4\textwidth]{EjemploCrudo01.png} \\
    \includegraphics[width=0.4\textwidth]{AbsLFP-cb-716.png} &
    \includegraphics[width=0.4\textwidth]{EjemploSuave01.png}
  \end{tabular}\\
 Sí, pero tratando los datos con \emph{delicadeza}. 
\end{frame}


\begin{frame}
\frametitle{Del Potencial Local a la DFC} 
\begin{center}
  \begin{tabular}{cc}
    \includegraphics[width=0.5\textwidth]{MapaLFPObienetall.png}   &
    \includegraphics[width=0.3\textwidth]{Decaimiento01.pdf}
  \end{tabular}
  \begin{tabular}{ccc}
    \includegraphics[width=0.3\textwidth]{LFP-cb-716.png} &
    \begin{tabular}{@{}c}
      ¿$I_m\propto -\nabla^2 \phi$? \\ 
      ¡ $  \partial_t \rho =\nabla \cdot \vec{J}$ !
    \end{tabular}
    &
    \includegraphics[width=0.3\textwidth]{CSD_SeminarioJunio-cb-732.png}    
   \end{tabular}
  \end{center}
{\small No completamente correcto. La aplicación del operador Laplaciano en este caso
nos da la \emph{``densidad instantanea de cargas'' }.}
\begin{flushright}
  {\tiny Obien \emph{et al.}, Frontiers in Neurosc. 8, 423 (2015)\\
         Bédard \emph{et} Destexhe, PRE 84, 041909 (2011) \\
         Moreland, Proc. Vis. Comp., 5876, 92  (2009) }
\end{flushright}
\end{frame}



\begin{frame}
\frametitle{Promedios vectoriales o \emph{Centro de Masa}} 
Tanto en la densidad de cargas, como en la densidad de fuentes de corriente
hay una distinción clara entre \emph{pozos} y \emph{fuentes}.
\begin{center}
  \begin{tabular}{cc}
    \includegraphics[width=0.25\textwidth]{DiagramaCAetDGCSDEspanol01.png}   &
    \includegraphics[width=0.35\textwidth]{CSD_SeminarioJunio-cb-732.png}
  \end{tabular}
\includegraphics[width=0.5\textwidth]{CSD_CM_SeminarioJunio-732.png}
\end{center}
  
\end{frame}

\begin{frame}
  \frametitle{To do List:}
  
  \begin{itemize}
  \item Estructura de datos que permita visualizar la trayectoria de los centro de masa.
  \item Caracterizar la aparición y desvanecimiento de estas trayectorias en actividad evocada.
  \item Ver si es posible caracterizar \emph{frentes de onda} para la propagación
    de la actividad. 

  \end{itemize}
  
\end{frame}



\end{document}
