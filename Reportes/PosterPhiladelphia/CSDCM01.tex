CSD establishes a difference that separates two sets in our space, namely, the set
of all sources and the set of all sinks. The zero set is the border between them.
We pick a \emph{numerical zero} that gives this set certain width (it becomes an
open set) and thus the separation betweeen the sources and sinks becomes more
pronounced. Moreover, this numerical zero has the effect of separating each
of these signed density sets into disjoint components that correspond
roughly to the expected poles of the neuronal activity. Is in this represantation
that we can apply then the concept of ``center of mass''. We calculate the
center of mass for each one of these disjoint components simultaneously, using
as ``mass'' the source or sink density. This vector averaging technique reveals a putative ``centre of influx/eflux'' which can be located in various parts of the hippocampus
frame by frame.

By concatenating probable succesors in time to each one of these centers, we can
trace how the activity displaced itself along neighbouring units. The plot of these
concatenated centers corresponds to the trayectory that the activity traces as
close lying neuronal units activate and deactivate. Our analysis permits us to
discern very fast changes on the location of the stronger activity and very small
deviations from the average locus  of active tissue. 
