Epileptic or epileptic-like activity is characterized by excessive synchronous action 
in the brain.  High density microelectronic arrays (HDMEAs) can be used to record
the generated LFP associated to this activity at hundreds or thousands of near
locations and at a  high frequency. These last years said devices are
approximating resolutions comparable to the neuron scales involved (of the
order 50 µm). However, comparing thousands of electro-physiological traces with dense
patterns can obstacle the discovery of the location, succession, and mode of
propagation of the activity.

An usual technique to discover such information is to plot ``heat maps'' of
the absolute diference in voltage of the LFP at each recording time,
and then animating the
color plots in succesion to observe the start, propagation, and end 
of the seizure.  This raw treatment can serve in general to ilustrate and to learn
the general characteristics of the seizure, but it is poor in resolution and
numerically vague. Here we propose a refination of the  methods used for analysing
propagation of activity in electrophysiological recordings. Our contribution permits
to separate various paths of symultaneous activity and track fine variations
between various almost synchronic units.

The method in question consists on extacting rigorous separable data of the CSD
representation and then applying a ``Center of Mass analysis'' on each set. 
This enables us to sepparate precisely units of
activity by their succesive activities. Our analyisis consist on four steps.

On the first step we transform the LFP data into CSD by a simple numerical
Laplacian operator. This representation of the data has some advantages over the LFP.
The most cited one is that the spread of the signal over area is reduced, pointing
more accurately to sites of activity. Also it represents, as it name suggest,
a density. That means that it can be used for obtaining weighted averages,
in particular, we use it to obtain vector averages of sinks and sources. Lastly, it
has a zero value with a precise meaning, separating sinks from sources (influx and
eflux sites). The zero set is a natural border that allows us to separate
the various poles that are formed by active units. This is important for the
second step of the analysis, in which we use this border to segment the sets of
sources and sinks into disjoint components, which correspond roughly to poles
of firing neurons. The third step is to calculate the ``center of mass'' of each
one of these components, frame by frame. As ``mass'' we use the CSD, which is, by
definition, a density, and then the center thus obtained is precisely the center of
mass for a negative/positive pole. At this step we have a collection of centers
of mass for both sinks and sources, each one encompasing its ``intensity'' (total
mass) and spatiotemporal coordinates. In the fourth step we try to concatenate the
centers of each frame with those of the next, trying to find the most probable
succesor to each one by distance and intensity. Every chain thus obtained
is a path, which reveals very fine and rapid displacements of the neuronal activity
on each site of the tissue. 


In the following figures we illustrate this ideas with data obtained from a
rat's hippocampus wich has beeen prepared with 4AP... bla bla Franco, check this.

