\documentclass[letterpaper, 12pt]{article}

\usepackage[utf8]{inputenc}
\usepackage[margin=5mm, landscape]{geometry}
\usepackage{graphicx}

\usepackage{caption}
\usepackage{subcaption}


\begin{document}

A comparison of the epileptic burst in LFP and CSD representations.
A) A snapshot in the middle of the epileptic activity, as measured in
all the electrodes of the BioCAM 4096. The color map has been chosen as to
minimize the effect noise, otherwise the data is presented as measured respect
to a grounded zero . Three co-linear sets  of electrodes along CA have been
marked for displaying other representations in the next graphs.
B) The electro-physiological traces at the selected nodes of graph A.
Notice that the scale has meaning, but we have omitted the zero axis as it
is arbitrary and only deflections have meaning.
C) The CSD representation of the same data. Here we choose a color map
center around zero (white) which has meaning, but the scale of densities is
in arbitrary units. Notice that the spread of the colored areas is smaller, but
the contrast is higher and we notice smaller  sources and sinks near
the stratum piramidale which are lost in A.
D) Also we present the traces of the activity as in B, but here in CSD
representation. In contrast, here the vertical scale is in arbitrary units,
and can be omitted, but the zero has meaning as it separates
sources from sinks, so we present it. Time scale is the same as in B). 

\end{document}
