\documentclass{article}

%\usepackage[utf8]{inputenc}
\usepackage{fontspec}
\usepackage{amsmath, amssymb}

\usepackage{graphicx}

\usepackage{caption}
\usepackage{subcaption}



\setmainfont[Ligatures=TeX]{Oldstyle HPLHS}

\newcommand{\Jd}{\mathbf{J}}
\newcommand{\EF}{\mathbf{E}}
\newcommand{\cond}{\boldsymbol{\sigma}}
\DeclareMathOperator{\diver}{div}
\DeclareMathOperator{\grad}{grad}

\title{CSDA and Vector Averaging tracking at the Cornu Amonis}
\author{Laboratorio 19, CINVESTAV}

\begin{document}

\maketitle

\begin{abstract}
Vector Averaging, also called sometimes ``Center of Mass Analysis'', has
been used to track centers of 
neuronal activity from electrophysiological recordings. 
Here we propose various refinaments of the idea and apply the techniques
to high density data recorded from the rat hipocampus. 
\end{abstract}


\section{Introduction}

Vector Averaging or Center of Mass techniques have been used to follow
the tracks or paths of neural activity in electrophysiological
measurements corresponding to different experiments 
\cite{ Abbot94, Chao05, Manjarrez07, Xiong07}. The central
idea is to regard recording cites as vectors of an Euclidian space,
and then calculate an average using the recorded data in some
way as the wheight of each vector. Starting with Abbott et Salinas 
\cite{Abbot94}, the usual ``mass'' where the spikes 
counted over a time window as their mass, and called their
vector average the Center of Activity (eg. \cite{Demas03, Chao05}). 
Manjarrez et al. \cite{Manjarrez07},
on the other hand, used the positive local field potential after choosing
an adecuate zero (ground) value.  
 


\section{Methods}

We record the LFP from slices of the rat's hipocampus bla bla bla.

The electrophysiological data is exported as a three dimensional array
containing the EEG recording at each lattice cite and at each instant
of time, measured in $\mu V$. We proced to obtain the CSD as follows.

The data is smoothed in the temporal dimension 
 with a Gaussian Low Pass filter, which
gets rid of the noice induced by the line feed. Then,  with a rotational
invariant Gaussian blur (a convolution with a Gaussian Matrix), we
smooth the data spatialy, as to minimize the cross efects due to the 
geometry of the measuring MEA. Due to the high spatial resolution 
that the BIOCam 4096, has, a discrete Laplacian operator is enough to
obtain the CSD, without need to use more sofisticated techniques such as
iCSDA \cite{Leski2011}. The better suited Discrete Laplacian is also the
one which minimizes the effect of the rectangular grid. It has been shown
by Lindberg \cite{Lindberg90} that the following expression gives the 
best approximation to a rotational invariant Laplacian:
\begin{equation}
\nabla^2_{disc}=(2/3)\begin{pmatrix}
0 & 1 & 0 \\
1 & -4 & 1 \\
0 & 1 & 0 
\end{pmatrix}+(1/3)
\begin{pmatrix}
1/2 & 0 & 1/2 \\
0 & -2 & 0 \\
1/2 & 0 & 1/2
\end{pmatrix} 
\end{equation}
The convolution with this matrix is well defined everywhere except at the
edges. There  we extend the nearest recording to obtain a reasonable 
aproximation for the CSD. 

The sets of all sources (converely, sinks) where the lattice points
with an CSD greater (lower) than 0.01 a.u. This threshold is 
well below the noice level of the data but high enough to adecuatelly
establish significant border between the sets.

Then we proceed to separate the disjoint components of each of the sets
using 8-neighbourhoods (each recording cite is considered to be connected
with its 8 closer neighbours) and a simple pass algorithm 
\cite{Vincent91}. Once that the list of disjoint components is complete,
we compute Vector Average for each component using the absolute value
of the CSD as mass. This produces a ``Center of Mass'' which is clearly
distinct for each activity cite and type and which is close to the
recording cites where the activity is measured. 

The next step is to concatenate the Center of Mass at each instant with
the most probable continuation at the next timestep, in order to obtain
a track of activity. 


\bibliographystyle{plain}
\bibliography{BiblioReportes01}



\end{document}
