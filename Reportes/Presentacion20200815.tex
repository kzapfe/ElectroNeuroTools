\documentclass[10pt]{beamer}

\usepackage[utf8]{inputenc}
\usepackage{mhchem} % para formulas quimicas

%\AtBeginSection[]{
%\begin{frame}
%\frametitle{Table of Contents}
%\tableofcontents[currentsection]
%\end{frame}}

\usetheme{Berlin}
\usecolortheme{beaver}
\usefonttheme{serif}

\begin{document}

\section{LFP and Multi-unitary Activity}

\begin{frame}
  \frametitle{LFP Measurements}
  {\small The LFP is the Electric Potential in
  the extracellular medium around neurons. It is generated
  by the transient ionic imbalances that are consequence
  of neuronal activity. LFP is regarded as the consequence
  of multi-unitary activity, and as such filtering
  is done to prevent single spikes from dominating the signal.
  We inquiry ourselves if from the LFP measurement we can
  track some form of functional connectivity between
  nearby multi-unitary circuits.}
\end{frame}


\begin{frame}
  \frametitle{LFP moves as a wave during epileptic attacks}
  Inspiration for this method was found in Epileptic Activity
  in the Hippocampus. In such cases, activity moves in continuous waves across
  the tissue, as opposed to jumping between sites. This suggest
  that there is effective coupling between nearby groups of
  neurons. The channels  can be  ephaptic coupling, integrated
  sub-threshold activity, or other media that help
  to generate synchronicity and direct communication between close
  circuits.
\end{frame}


\begin{frame}
  \frametitle{LFP Inconveniences}
  Unfiltered measurements of the extracellular potential
  have some drawbacks:
  \begin{itemize}
  \item It spreads as $1/d$, but not in an homogeneous fashion.
  \item It has considerable noise.
  \end{itemize}
    \begin{center}
  \includegraphics[width=0.9\textwidth]{LFPVariosTrazosDeEjemplo01.png}  
  \end{center}

\end{frame}


\section{CSD}

\begin{frame}
  \frametitle{The CSD representation}
  A formally equivalent representation of the Electrical Activity
  as seen from the extracellular space is the CSD.
  Ideally, it is the negative Laplacian of the LFP.
  In practice it implies some filtering, as differential operators
  are highly sensible to noise. Before we apply some numerical
  difference operators, we smooth the data with the following procedures:
  \begin{itemize}
  \item A Gaussian Blur: spatial noise filtering.
  \item A Gaussian Convolution: temporal smooth low pass filter. 
  \end{itemize}
\end{frame}

\begin{frame}
  \frametitle{The CSD itself}
  Then a finite difference operator can be applied.
  For data in a rectangular lattice, the closest one
  to rotational invariance is the Lindberg operator:
  \begin{equation}
    \nabla^2_{1/3}=2/3
    \begin{pmatrix}
      0 & 1 & 0 \\
      1 & -4 & 1 \\
      0 & 1 & 0
    \end{pmatrix}
    +1/3
    \begin{pmatrix}
      0.5 & 0 & 0.5 \\
      0 & -2 & 0 \\
      0.5 & 0 & 0.5
    \end{pmatrix}  
\end{equation}
 This operator is convoluted with the data frame by frame. 
\end{frame}

\begin{frame}
  \frametitle{Comparision between CSD and LFP}
  \begin{center}
  \includegraphics[width=0.75\textwidth]{PrimerCuadro-01.png}
\end{center}

\end{frame}

\begin{frame}
  \frametitle{Advantages of CSD representation}
  \begin{itemize}
  \item It reduces spread.
  \item It separates the signal into two
    \emph{physiologically interpretable} sets of actions.
  \end{itemize}
  The second point is the one that we shall exploit. 
\begin{center}
  \includegraphics[width=0.85\textwidth]{MapaLFPObienetall.png}
\end{center}
\end{frame}


\begin{frame}
  \frametitle{Sinks and Sources}
  As a neuron prepares itself for activity, it begins
  to absorb \ce{Na+} ions of the outside medium. This creates a
  \emph{``Sink''} in the volume surrounding the base of the axon.
  Almost immediately, the neurons spit out \ce{Ka+} ions, producing a
  \emph{``Source''} of current outside the cell. The different localisation
  of these paired actions create a dipole pattern in the CSD representation.
  The CSD representation gives us a meaningful distinction
  between phases of the action.
\end{frame}


\begin{frame}
  \frametitle{Sinks and Sources Separate The Activity}
  A Source is a Source above certain threshold and a
  Sink is a Sink below certain threshold. Everything
  in between is regarded as noise.

  We determine this threshold by an heuristic criterion.
  In this example we show the borders done with $2\sigma$
  of the noise signal.
  \begin{center}
    \includegraphics[width=0.5\textwidth]{EjemploFronteraComponentesDisjuntos01.png}
  \end{center}
  

\end{frame}

\begin{frame}
  \frametitle{Separated activity results in Disconnected Sites.}
  By vector averaging we can find the Center of each one of
  these sites.\\
  Our knowledge about the CSD representation guaranties us that
  the patches are going to be well localizated over the active sites.
  Also, by choosing a high enough treshold value, we guarantee that
  the patches are going to be over different active multi unitary groups.
  \begin{center}
    \includegraphics[width=0.4\textwidth]{CSDCM-550.png}
  \end{center}
  
\end{frame}

\section{Trajectories}

\begin{frame}
  \frametitle{Concatenate the Centers}
  Also, by continuity, we shall expect that the obtained CM move
  to nearby sites as time pases, their displacement revealing the
  slight asyncronicity between units of an integrated circuit.
  \begin{center}
    \includegraphics[width=0.9\textwidth]{TresCuadrosConTrayectoria01.png}
  \end{center}
  
\end{frame}

\begin{frame}
  \frametitle{Trajectories}
  Once the trayectories are obtained, a series of graphical and
  numerical analysis are possible. 
  \begin{center}
    \includegraphics[width=0.8\textwidth]{PathsComparaTraub.png}
  \end{center}
  
\end{frame}

\begin{frame}
  \frametitle{Velocities of the CMs}
  We have obtained the velocities of these centers.
  These are well within what is expected from axonal transmission
  of potentials of action.
  \begin{center}
    \includegraphics[width=0.5\textwidth]{velocitiesSinksandSources.png}
  \end{center}
\end{frame}

\begin{frame}
  \frametitle{Other interesting relations}
  A curious representation of an Epilleptic Wave.
  \begin{center}
    \includegraphics[width=0.4\textwidth]{IntensityAgainstMoment.png}
  \end{center}
  Every trace in the above plot is  CSD. The circular dot is its Average
  Intensity at half of its duration. The $x$ axis is in seconds, and
  the $y$ axis is charge per area in arbitrary units. 
\end{frame}

\section{What else?}

\begin{frame}
  \frametitle{An observation}
  Trajectories are never \emph{too long}. Gaps are present
  in what appears to be a systemic disposition.
  \begin{center}
    \begin{tabular}{cc}
      \includegraphics[width=0.4\textwidth]{dbscanposejem01.png}
      &
      \includegraphics[width=0.3\textwidth]{zipfexample.png}
    \end{tabular}
  \end{center}
\end{frame}



\begin{frame}
  
\end{frame}


\end{document}
