\documentclass{beamer}


\usepackage{amsmath}
\usepackage[utf8]{inputenc}
\usepackage[spanish]{babel}
\usepackage{color}

\decimalpoint

\usefonttheme{serif}

\newcommand{\dd}{\, \mathrm{d}}
\newcommand{\xq}{\, \mathbf{x}}

\begin{document}


\begin{frame}
  \frametitle{De la actividad evocada a las trayectorias de los CM}
  Dado que la actividad evocada produce respuestas
  \emph{perfectamente} acompazadas, podemos
  promediar sobre varios eventos experimentales
  a fin de tener una señal más limpia. 
  \begin{equation}
    \varphi(\xq)
    =\int \frac{1}{4\pi\epsilon}
    \frac{\rho(\xq')}{|\xq-\xq'|} \dd^3 \xq'  
  \end{equation}
  \begin{center}
  \begin{tabular}{ccc}
   \textcolor{red}{LFP} & 
   \includegraphics[width=0.4\textwidth]{PotencialEvocado01.pdf} &
   \textcolor{blue}{CSD}
  \end{tabular}
  \end{center}
\end{frame}


\begin{frame}
\frametitle{Del Potencial Local a la Densidad de Fuentes de Corriente} 
El operador Laplaciano formalmente invierte la expresión para
el potencial eléctrico (medio uniforme e isotrópico).
\begin{equation}
  -\nabla^2\phi \propto \rho
\end{equation}
\begin{equation}
I=\partial_t \rho= - \rho \cdot \sigma/\epsilon 
  \end{equation}
\ldots pero el op. Laplaciano es una idealización:
\begin{equation}
  \nabla^2 f (\xq) =\lim_{\varepsilon \to 0}
  \frac{\sum_k f(\xq+\varepsilon_k)+ f(\xq-\varepsilon_k)-2 f(\xq)}
       {\varepsilon^2}  
\end{equation}
\begin{flushright}
  {\tiny Nicholson \emph{et} Freeman , J.   Neurophys. 38, 356 (1975)\\
         Bédard \emph{et} Destexhe, PRE 84, 041909 (2011) }
\end{flushright}

\end{frame}

\begin{frame}
  \frametitle{Un Laplaciano Numérico}
  \begin{equation}
    \nabla^2_{n}=2/3
    \begin{pmatrix}
      0 & 1 & 0 \\
      1 & -4 & 1 \\
      0 & -4 & 0    
    \end{pmatrix}
    +1/3
    \begin{pmatrix}
      0.5 & 0 & 0.5 \\
      0 & -2 & 0 \\
      0.5 & 0 & 0.5    
    \end{pmatrix}
  \end{equation}
  \begin{center}
  \begin{tabular}{cc}
    \includegraphics[width=0.4\textwidth]{LFP-cb-716.png} &
    \includegraphics[width=0.45\textwidth]{CSD_SeminarioJunio-712.png}    
  \end{tabular}
  \end{center}
\begin{flushright}
  {\tiny Lindberg, PAMI, 90, 234 (1990)}   
\end{flushright}
\end{frame}

\begin{frame}
  \frametitle{Una aproximación \emph{inversa}: iCSDA, kCSDA}
    \begin{align}
    \varphi(\xq)
    & =\int \frac{1}{4\pi\sigma}
    \frac{I(\xq')}{|\xq-\xq'|} \dd^3 \xq' \\
    & \approx \sum_j \frac{1}{4\pi\sigma}
    \frac{I(\xq_j)}{|\xq-\xq_j|} |\Delta \xq|     
  \end{align}
    Esto puede ponerse como una transformación lineal:
\begin{equation}
    \phi_k=\sum_j F_{jk} I_j
\end{equation}
Donde $F$ contiene suposiciónes sobre la distribución, e.g.
\begin{equation}
  F_{jk}=\iiint_{\xq_k} \dd x \dd y \dd z
\frac{H(z)}{4\pi\sigma \sqrt{(x-x_j)^2+(y-y_j)^2+z^2}} 
\end{equation}
\end{frame}

\begin{frame}
  \frametitle{iCSDA, kCSDA II}
  \ldots y sólo tendriamos que invertir $F$ para obtener $I$
  \begin{equation}
    I_m=F^{-1}\phi 
  \end{equation}
  Esto requiere resolver $n\times n$ integrales, cada una con
  $n\times n$ contribuciones, y luego invertir la matriz, que es una operación
  que crece como $n^3$ \ldots
  Si nos va bien, esta operación crece como $O(n^7)$.

  Resuelta \emph{en principio}.

 \begin{flushright}
  {\tiny  Łesky \emph{ et all}, Neuroinform, 9, 401 (2011) \\
      Pettersen \emph{et all} , JNM  154, 116 (2011) \\
      Potworowsky \emph{et all}, Neur. Comp. 24, 541 (2012). }
  \end{flushright}
\end{frame}



\begin{frame}
\frametitle{Un cero bien localizado.} 
Tanto en la densidad de cargas, como en la densidad de fuentes de corriente
hay una distinción clara entre \emph{pozos} y \emph{fuentes}.
No es el caso para el LFP, que mide las cosas a partir de una ``tierra''
arbitraria. 
\begin{center}
    \includegraphics[width=0.75\textwidth]{CSD_SeminarioJunio-cb-732.png}
\end{center}
\end{frame}

\begin{frame}
  \frametitle{Promedios Vectoriales o \emph{Centros de Masa}}
  \begin{equation}
    \langle \xq(t) \rangle=\sum_k m(\xq_k)\xq/\sum_k m(\xq_k)
  \end{equation}
  
  \begin{center}
    \includegraphics[width=0.75\textwidth]{ImagenManjarrez01.jpg}
  \end{center} 
  
 \begin{flushright}
  {\tiny  Manjarrez \emph{ et all}, Brain Research, 1145, 239 (2007) \\
      Chao \emph{ et all}, Neuroinformatics, 3, 263 (2005) }
  \end{flushright}

\end{frame}


\begin{frame}
\frametitle{Promedios vectoriales o \emph{Centro de Masa}} 
Tanto en la densidad de cargas, como en la densidad de fuentes de corriente
hay una distinción clara entre \emph{pozos} y \emph{fuentes}.
\begin{center}
  \begin{tabular}{cc}
    \includegraphics[width=0.35\textwidth]{DiagramaCAetDGCSDEspanol01.png}   &
    \includegraphics[width=0.55\textwidth]{CSD_CM_SeminarioJunio-732.png}
  \end{tabular}
\end{center}  
\end{frame}


\begin{frame}
\frametitle{Centros de Masa en el espacio de las CSD.} 
Tanto en la densidad de cargas, como en la densidad de fuentes de corriente
hay una distinción clara entre \emph{pozos} y \emph{fuentes}.
\begin{center}
  \begin{tabular}{cc}
    \includegraphics[width=0.35\textwidth]{DiagramaCAetDGCSDEspanol01.png}   &
    \includegraphics[width=0.55\textwidth]{CSD_CM_SeminarioJunio-732.png}
  \end{tabular}
\end{center}  
\end{frame}


\begin{frame}
\frametitle{Centros de Masa en el espacio de las CSD.} 
Tanto en la densidad de cargas, como en la densidad de fuentes de corriente
hay una distinción clara entre \emph{pozos} y \emph{fuentes}.
\begin{center}
  \begin{tabular}{cc}
    \includegraphics[width=0.35\textwidth]{DiagramaCAetDGCSDEspanol01.png}   &
    \includegraphics[width=0.55\textwidth]{CSD_CM_SeminarioJunio-732.png}
  \end{tabular}
\end{center}  
\end{frame}

\begin{frame}
\frametitle{Trayectorias de Actividad} 
El seguimiento continuo de los CM.
\begin{center}
    \includegraphics[width=0.7\textwidth]{TrayectoriasLimpiasNumeros01.png} 
\end{center}  
\end{frame}


\end{document}
